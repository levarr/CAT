\documentclass[a4paper]{report}

\usepackage[margin = 1in]{geometry}
\usepackage[utf8]{inputenc}
\usepackage[T1]{fontenc}
\usepackage{amsthm, amsfonts, amsmath, amssymb}
\usepackage{tcolorbox}
\usepackage{capt-of, subcaption}
\usepackage[figurename=Fig.]{caption}
\usepackage[font=small,labelfont=bf]{caption}
\usepackage{textcomp}
\usepackage{wrapfig}
\usepackage{hyperref}
\usepackage{fancyhdr}
\usepackage{float}
\usepackage{multirow}
\usepackage{mathtools}
\usepackage{physics}
\usepackage{cleveref}

\graphicspath{ {../attachments/} }

\setlength\parindent{0pt}

\renewcommand{\partname}{Parte}
\renewcommand{\contentsname}{Indice dei contenuti}
\renewcommand{\chaptername}{Capitolo}

\newcommand{\starbreak}{%
    \begin{center}
        $\ast$~$\ast$~$\ast$
    \end{center}
}

\newcommand{\bb}{
	\bigbreak
}

\newcommand{\R}{
	\ensuremath
	\mathbb{R}
}

\newcommand{\llimit}[3]{
	\ensuremath
	\lim_{#1 \rightarrow #2} #3
}

\newcommand{\pinf}{
	\ensuremath
	+\infty
}

\newcommand{\minf}{
	\ensuremath
	-\infty
}

\newcommand{\intreal}[2]{
	\ensuremath
	\int_{\R} #1 \dd{#2}
}

\newcommand{\intnoreal}[4]{
	\ensuremath
	\int_{#1}^{#2} #3 \dd{#4}
}

\newcommand{\ifincases}[1]{
	\ensuremath
	\; \textrm{ se } \; #1
}

\newcommand{\sgn}[1]{
	\operatorname{sgn} \left( #1 \right)
}

\newcommand{\paren}[1]{
	\ensuremath
	\left( #1 \right)
}

\newcommand{\sparen}[1]{
	\ensuremath
	\left[ #1 \right]
}

\newcommand{\resource}[3]{
	\begin{center}
	\includegraphics[scale = #1]{#2}
	\end{center}
	\captionof{figure}{#3}
}

\newcommand{\rarr}{
	$\longrightarrow$
}

\tcbuselibrary{theorems, breakable, skins}

\definecolor{col1}{HTML}{FF7878}
\definecolor{col2}{HTML}{51B5F8}
\definecolor{col3}{HTML}{68E1AA}
\definecolor{col4}{HTML}{B869EA}
\definecolor{col5}{HTML}{FF5500}

\newtcbtheorem[no counter]{defin}{}% environment name
      {
      %attach boxed title to top center={yshift=-3.5mm},
      colback=col4!10, colframe=col4, colbacktitle=col4!75, coltitle=black,
      fonttitle=\bfseries,
      sharp corners=all,
      breakable,
      separator sign none,
      }%
{defin}% label prefix

%\newtheorem{them}{Teorema}[section]
\newtheorem{esem}{Esempio}[section]
%\newtheorem{eser}{Esercizio}[chapter]

\pagestyle{fancy}
\fancyhf{}
\fancyhead[EL]{\nouppercase\leftmark}
\fancyhead[OR]{\nouppercase\rightmark}
\fancyhead[ER,OL]{\thepage}
\fancyfoot[EC,OC]{\nouppercase\leftmark}

\renewcommand{\headrulewidth}{1pt}
\renewcommand{\footrulewidth}{1pt}

\usepackage{rotating}
\usepackage{cancel}
\usepackage[framed]{matlab-prettifier}


\title{
    \textbf{Appunti di Controlli Automatici T} \\
       \large Alma Mater Studiorum - Università di Bologna \\
    \large Facoltà di Ingegneria Informatica - 9254
}
\author{\textbf{Fabio Colonna} - fabcolonna@icloud.com, \\
        fabio.colonna3@studio.unibo.it }
\date{AA 2022-23}

\makeindex
\begin{document}

\maketitle
\tableofcontents
\newpage

\part{Introduzione}

\section{Idea sulla disciplina}

I Controlli Automatici (CA da ora in poi) rappresentano la disciplina che ha come obiettivo primario quello di \textbf{sostituire l'intelligenza unama con un sistema automatico - intelligenza artificiale - in modo da permettere ad una macchina manuale di poter operare in autonomia.} Possiamo chiamare questo processo \textbf{automazione}. Faremo largo uso di modelli matematici per definire con precisione i parametri necessari di cui abbiamo bisogno per generare effettivamente questo comportamento. Le applicazioni di queste tecnologie nel mondo reale sono pressoché infinite al giorno oggi: automotive (si pensi alla guida autonoma, ma anche a sistemi più "semplici" quali il controllo di trazione o l'ESP), apparecchi elettrodomestici, ambito militare (droni, autopilot nei veivoli)...
\bb
Il corso sarà diviso in due parti. La prima, cosiddetta di \textbf{analisi}, verterà su come \textbf{modellare un sistema fisico} utilizzando strumenti matematici per capirne il suo comportamento. Questo può essere fatto guardando il \textbf{comportamento dell'uscita al variare dei parametri in ingresso} del sistema suddetto. La seconda parte, chiamata \textbf{sintesi}, fornirà i metodi per la \textbf{progettazione di un sistema di controllo}, i.e. un \textbf{sistema + controllore}, cioè il dispositivo che a tutti gli effetti implementerà l'automazione, comandando il macchinario (il sistema) al posto nostro. Il punto di vista sarà diverso: \textbf{capire di quali ingressi abbiamo bisogno per generare il comportamento che vogliamo, i.e. l'uscita desiderata.}

\section{Terminologia ed elementi costitutivi}

\begin{defin}{Terminologia}{}
	\begin{enumerate}
		\item \textbf{Sistema} \rarr oggetto/fenomeno fisico per il quale si vuole ottenere un \textit{comportamento desiderato}. Esempi sono: impianti industriali, bracci robotici, veicoli/veivoli... Esso è composto da un \textbf{ingresso}, al quale arriva una grandezza che tipicamente \textbf{modifica il comportamento}, e un'\textbf{uscita}, che rappresenta il suddetto comportamento ed è tipicamente monitorata da \textit{sensori.}
		\item  \textbf{Controllore} \rarr unità di calcolo che determina \textbf{l'andamento delle variabili di ingresso}, in modo tale da generare l'uscita desiderata. A livello analitico questa entità è un sistema di equazioni matematiche.
		\item \textbf{Sistema di controllo} \rarr rappresenta la coppia \textbf{sistema + controllore}, ed è quello che il procedimento di sintesi avrà come obiettivo. 
	\end{enumerate}
\end{defin}


\resource{0.3}{sistema}{Schematizzazione di un sistema}
\bb


Un uomo che guida un'automobile è un esempio di sistema di controllo: l'auto è il sistema, il cervello umano è il controllore. L'uscita del sistema, i.e. la sua velocità o direzione, è monitorata da sensori naturali quali la vista, l'udito, ma anche da artificiali (ad es. il tachimetro). Sulla base di tutti questi dati, il cervello manda in ingresso all'auto un nuovo set di ingressi (angolo di sterzatura, pressione sul pedale acceleratore o freno) in modo da generare il comportamento che vuole.
\newpage
A partire da questo esempio, è possibile introdurre le due principali

\begin{defin}{Tipologie di controllo}{a}
\begin{enumerate}
	\item \textbf{in anello aperto (feedforward)} \rarr il controllore comanda il sistema mandandogli \textbf{ingressi valutati guardando solo il segnale di riferimento};
	\item \textbf{in anello chiuso (feedback)} \rarr oltre a guardare il segnale di riferimento, il controllore \textbf{utilizza anche i dati che istante per istante riceve per mezzo dei sensori.} Questo approccio è senz'altro quello predominante nel mondo dei CA.
\end{enumerate}
\end{defin}

\resource{0.80}{feedback}{Esempio di funzionamento feedback. Notare il segnale che rientra nel controllore (dati dei sensori)}

\section{Progetto di un sistema di controllo}

L'approccio che porta al progetto final e è più o meno il seguente:
\begin{itemize}
	\item \textbf{Definizione delle specifiche} \rarr comportamento desiderato del sistema, costo, indice di performance... Sono tutte informazioni tipicamente vincolate, ossia fornite dal datore di progetto e non modificabili dall'ingegnere;
	\item \textbf{Modellazione del sistema} \rarr eseguito tipicamente con l'aiuto di figure specialistiche del settore nel quale il macchinario poi sarà destinato a lavorare. Ha come obiettivo quello di ricercare modelli aventi un giusto compromesso tra complessità/semplicità nella sintesi, definire gli ingressi e le uscite, procedere con la codifica ed eventualmente validarla mediante simulazioni. Molto spesso si parla di \textit{modello di controllo}, tipicamente di complessità relativamente bassa, e \textit{modello di test}, estremamente più complesso in quanto creato al computer e reso quanto più verosimile possibile;
	\item \textbf{Analisi del sistema} \rarr Eseguito per studiare le \textit{proprietà strutturali} del modello, nonché capirne le sue capacità in modo da trovargli applicazioni opportune;
	\item \textbf{Sintesi della legge di controllo} \rarr A seguito di considerazioni fatte sulla \textit{scelta degli elementi tecnologici} da utilizzare per la realizzazione (device di elaborazione, elettronica di acquisizione/attenuazione, sensori/attuatori...), nonché una fase di sperimentazione eseguibile mediante numerosi approcci (ad es. HW in the loop, in base all quale al prototipo virtuale si inviano controlli provenienti da \textit{schede di controllo reali}), si genera il sistema di controllo finale automatizzato secondo le specifiche.
\end{itemize}

\part{Analisi nel dominio dei tempi}
\chapter{Sistemi dinamici in forma di stato}
Iniziamo adesso ad approcciare la materia in modo più matematico. Consideriamo il seguente circuito: abbiamo un generatore di tensione $v_G$ che può variare liberamente, e che quindi costituisce l'\textbf{ingresso}, e la tensione del condensatore $v_C$ che invece non è attivamente modificabile, dunque rappresenta lo \textbf{stato interno del sistema}. Possiamo esprimere \textbf{la variazione dello stato interno in funzione di altre grandezze, tra cui l'ingresso.} Vediamo come:
\bb
\begin{minipage}{0.4\textwidth}
\resource{0.6}{circ1}{}
\end{minipage}
\begin{minipage}{0.6\textwidth}
\begin{equation*}
	LKT \rightarrow v_G-v_R-v_C=0
\end{equation*}
ma abbiamo $v_R=iR$ e, dall'equazione caratteristica dei condensatori, $i=C \dot v_C$. Combinando le due, abbiamo:
\begin{equation*}
	v_G-iR-v_C=0 \rightarrow v_G-RC\dot v_C-v_C=0
\end{equation*}
da cui la relazione finale, che lega la \textit{derivata prima di una grandezza, con le altre}:
\begin{equation*}
	\dot v_C(t)=\frac{1}{RC}\paren{v_G(t)-v_C(t)}
\end{equation*}
\end{minipage}
\bb
Procediamo ora chiamando la variabile di stato $x(t)$, e la variabile di ingresso $u(t)$. L'equazione diventa:
\begin{equation*}
	\dot x(t) = -\frac{1}{RC}x(t)+\frac{1}{RC}u(t).
\end{equation*}
Abbiamo però definito un sistema come una terna (ingresso, stato, uscita), per cui \textbf{fissiamo una grandezza che ci misura l'uscita del sistema}. Questa è assolutamente arbitraria, a differenza della variabile di stato. Prendiamo ad esempio $v_R(t)$. Chiamiamo questa $y(t)$. Per definizione di sensore, riusciamo a leggere ora la tensione ai capi della resistenza \textit{istante per istante. }Abbiamo:
\begin{equation*}
	y(t) = v_R(t) =i(t)R = RC\dot v_C(t)=RC\dot x(t)=-x(t)+u(t).
\end{equation*}
Definiamo adesso un modello finale, composto da \textbf{un'equazione che determina la variazione dello stato del sistema $x(t)$ e una che determina, istante per istante, l'uscita $y(t)$ del sistema:}
\begin{equation*}
\begin{dcases*}
	\dot x(t) = -\frac{1}{RC}x(t)+\frac{1}{RC}u(t) \\
	y(t) = -x(t)+u(t)
\end{dcases*}
\end{equation*}
Questa equazione è importante: abbiamo appena descritto un generico sistema fisico utilizzando un modello matematico di questo tipo, denominato \textbf{forma di stato}. Vediamone finalmente la definizione generale.

\newpage
\begin{defin}{Descrizione in \textit{forma di stato} di un sistema}{forma_di_stato}
Per sistemi continui, il tempo $t \in \R$. Diciamo che un sistema dinamico è \textbf{descritto in forma di stato se per esso abbiamo definito la seguente coppia di equazioni}:
\begin{equation}
	\begin{dcases}
		\dot x(t) = f\paren{x(t), u(t), t} \quad \textrm{equazione di stato (ODE I ordine)}\\
		y(t) = h(x(t), u(t), t) \quad \textrm{ equazione di uscita}
	\end{dcases}
\end{equation}
\begin{itemize}
	\item $x(t) \in \R^n$ è la variabile che rappresenta lo \textbf{stato del sistema all'istante $t$}, non direttamente visualizzabile all'esterno \textbf{a meno che non sia direttamente collegato con l'uscita} (la quale è costantemente monitorata mediante sensori, come già detto); 
	\item $u(t) \in \R^m$ è l\textbf{'ingresso del sistema all'istante $t$}, modificabile dall'esterno;
	\item $y(t) \in R^p$ è l'\textbf{uscita del sistema all'istante t}, monitorata da $p$ sensori. 
\end{itemize}
Questo modello matematico è in generale tanto più potente quanto più generale lo riusciamo a mantenere. Il problema a cui però andiamo incontro è la \textit{difficoltà crescente nel calcolo di una soluzione}, che vorremmo mantenere entro certi limiti (eventualmente computazionali).
\end{defin}

Dalla definizione si evince che la rappresentazione in forma di stato è solo in casi particolari di tipo scalare. Il più delle volte, specie all'aumentare della complessità del sistema che si vuole descrivere, si tratta di un sistema di equazioni multidimensionali. In particolare, alla luce delle dimensioni delle tre variabili principali, abbiamo che:
\begin{equation*}
	f:\R^n \times \R^m \times \R \rightarrow \R^n \quad \quad \quad h: \R^n \times \R^m \times \R \rightarrow \R^p
\end{equation*}
Notiamo inoltre che per ogni componente del vettore di $\dot x(t)$ la funzione $f$ dipende da \textbf{tutte le componenti del vettore di $x(t)$}.  Analogo discorso per il vettore $u(t)$. Questo diventa evidente se riscriviamo tutto in forma matriciale:

\begin{defin}{}{}
	\begin{equation*}
		x(t)= \begin{bmatrix}
			x_1(t) \\
			\vdots \\
			x_n(t)
		\end{bmatrix}
		\quad \quad \quad
		\dot x(t)= \begin{bmatrix}
			\dot x_1(t) \\
			\vdots \\
			\dot x_n(t)
		\end{bmatrix}
		\quad \quad \quad
		u(t)= \begin{bmatrix}
			u_1(t) \\
			\vdots \\
			u_m(t)
		\end{bmatrix}
		\quad \quad \quad
		y(t)= \begin{bmatrix}
			y_1(t) \\
			\vdots \\
			y_p(t)
		\end{bmatrix}
	\end{equation*}
	da cui le due equazioni (notare che all'interno di ciascuna $f_j$ o $h_j$ compaiono le intere matrici):
	\begin{equation}
	\label{f_h_matrices}
			\begin{bmatrix}
			\dot x_1(t) \\
			\vdots \\
			\dot x_n(t)
		\end{bmatrix} = \begin{bmatrix}
	f_1\paren{x(t),u(t), t} \\
		\vdots \\
			f_n\paren{x(t), u(t), t}
		\end{bmatrix} \quad \quad \quad 
		\begin{bmatrix}
			y_1(t) \\
			\vdots \\
			y_p(t)
		\end{bmatrix} =
		\begin{bmatrix}
	h_1\paren{x(t),u(t), t} \\
		\vdots \\
			h_p\paren{x(t), u(t), t}
		\end{bmatrix}
	\end{equation}
\end{defin}

Facciamo ora una valutazione più precisa in termini dimensionali. Partiamo innanzitutto col dire che, almeno in questo corso, sarà raro trovare dei sistemi espressi in forma di stato in cui le funzioni $f,h$ sono messe in relazione con $t$. Rimane comunque opportuno specificarla in quando è possibile che ci siano \textit{grandezze che non siano nè stato, nè ingresso} che dipendano dal tempo (si pensi ad una capacità $C(t)$ di un condensatore). Per quanto riguarda le relazioni dimensionali fra le varie grandezze, tipicamente:
\begin{equation*}
	\boxed{p \leq m} \quad \textrm{i.e. numero di uscite $\leq$ numero di ingressi (ragione di costi)},
\end{equation*}
\begin{equation*}
	\boxed{m \leq n} \quad \textrm{i.e. numero di ingressi $\leq$ numero di stati}.
\end{equation*}
In particolare, quando $m < n$ si parla di \textbf{sistema sotto-attuato}, \textbf{sovra-attuato} nel caso opposto, e \textbf{fully-actuated} se $m=n$ (numero di ingressi/attuatori = numero di gradi di libertà). Infine, un caso particolare avviene per i \textbf{sistemi single-input-single-output (SISO)}, in quanto si avrà:
\begin{equation*}
	n>1 \wedge m=p=1.
\end{equation*}
\bb
Abbiamo visto che l'equazione di stato è una ODE di primo ordine, per la quale nativamente esistono infinite funzioni risolutrici. Questo cambia se introduciamo uno \textbf{stato iniziale conosciuto, calcolato ad un istante iniziale} del tipo $x(t_0)=x_0$, in quanto potremmo a quel punto parlare (sotto opportune ipotesi di regolarità di $f$) di un vero e proprio \textbf{problema di Cauchy.} \textbf{Quando ci troviamo in queste condizioni, e vale $u(\tau), \tau \geq t_0$, si parla di sistema causale}, ossia di sistema la cui evoluzione è calcolabile guardando la storia passata del suo stato. (Da TLC sappiamo che si parla invece di sistema \textit{algebrico} quando il suo stato dipende solo dal presente, e non da quanto successo in precedenza). In nessun caso si ha un sistema (che sia reale) dipendente da valori futuri.
\bb
Il numero di variabili di stato dipende ovviamente dal problema fisico che stiamo analizzando, ma in generale può essere \textbf{grosso modo} associato al \textbf{numero di grandezze che compaiono derivate nel tempo} in fase di scrittura delle equazioni del sistema. Potrebbero esserci dei casi in cui la relazione tra VDS e sua derivata non sia di primo grado; in quei caso avremmo una ODE di un ordine necessariamente maggiore del primo. Per ovviare a ciò è opportuno introdurre delle VDS aggiuntive (che di conseguenza andranno ad aumentare la dimensione dello spazio di $x(t)$.
\bb
Nei sistemi dinamici \textit{discreti}  il tempo $t$ non si muove in un dominio continuo, ma è scandito da interi (appunto, discretizzato). La descrizione in forma di stato conterrà, di conseguenza, una ODE a differenze finite (anche detta FDE).

\bb
Gli esempi sulla scrittura di un sistema fisico in forma di stato sono su carta (carrello, auto in moto rettilineo, pendolo).

\section{Traiettoria ed equilibrio}
\begin{defin}{Traiettoria}{traiettoria}
Supponiamo di avere un sistema dinamico \textit{forzato} in forma di stato definito mediante un problema di Cauchy, quindi avente anche stato iniziale specificato:
\begin{equation*}
	\begin{dcases}
		\dot x(t) = f(x(t), u(t), t) \\
		y(t) = h(x(t), u(t), t)
	\end{dcases}, \quad x(t_0)  =x_0
\end{equation*}
È detta \textbf{traiettoria (o moviemento) del sistema} la funzione del tempo 
\begin{equation}
\paren{x(t), u(t)}, \ t \geq t_0
\end{equation}
che \textbf{risolve l'equazione di stato $\dot x(t)$}. In particolare, questa sarà la \textit{traiettoria di stato.} La funzione (della stessa tipologia) che soddisfa l'equazione dell'uscita $y(t)$ è invece detta \textit{traiettoria di uscita.} 
\bb
Senza l'ipotesi di forzatura, la traiettoria dipenderebbe \textit{solo dallo stato iniziale}: $(x(t)), \ t \geq t_0$.
\end{defin}

Dunque, il concetto di traiettoria è legato strettamente a quello di soluzione. Preso un generico sistema \textit{tempo invariante} (per semplicità), con condizione iniziale:
\begin{equation*}
	\begin{dcases}
		\dot x(t) = f(x(t), u(t)) \\
		y(t) = h(x(t), u(t))
	\end{dcases}, \quad x(t_0) = x_0
\end{equation*}
abbiamo che la \textit{traiettoria di stato} è quella funzione che risolve il problema di Cauchy dato dall'equazione di stato e dalla condizione iniziale. In simboli:
\begin{equation*}
	\paren{\bar x(t), \bar u(t)}, \ t\geq t_0 \quad \textrm{traiettoria} \quad \leftrightarrow \quad \begin{dcases}
		\dot{\bar x}(t) = f(\bar x(t), \bar u(t)) \\
		\bar x(t_0) = x_0
	\end{dcases}.
\end{equation*}

\begin{esem}
	Calcoliamo la traiettoria di stato del sistema dinamico avente questa scrittura di stato:
	$$
	\begin{dcases}
	\dot x(t) = u(t) = 1 \\
	x(t_0)=10	
	\end{dcases}
	$$
	Si tratta di un sistema forzato (sono presenti ingressi), per cui la traiettoria sarà necessariamente una funzione a sua volta dipendente dallo stato, quindi scrivibile come $\paren{x(t), u(t)}.$
	Integrando entrambi i membri:
	\begin{equation*}
		x(t) - x(t_0)=\int_{t_0}^{t}u(\tau) \dd{\tau} \rightarrow x(t) = x(t_0)+1(t-t_0) \rightarrow x(t) = 10 + t - t_0.
 	\end{equation*}
 	Abbiamo che la funzione $\paren{x(t), u(t)} = (10 + t-t_0, 1)$ è traiettoria di stato del sistema. Infatti, derivando, si riottiene 1.
\end{esem}

\begin{esem}
	Valutare la traiettoria dato il problema di Cauchy:
	\begin{equation*}
		\begin{dcases}
			\dot x(t) = ax(t)+bu(t) \\
			x(0)=10
		\end{dcases},
		\ con \ u(t) = \bar u \ \forall t \geq 0
	\end{equation*}
	Siamo davanti ad un'equazione differenziale ordinaria lineare non omogenea di primo ordine. La formula risolutiva (valida anche per le omogenee) è la seguente:
	\begin{equation}
		\boxed{\begin{dcases}
			\dot x(t) + a_0(t)x(t)=b(t) \\
			x(t_0) = x_0
			\end{dcases} \quad \longrightarrow \quad x(t) = e^{-A(t)} \sparen{x_0 + \int_{t_0}^t b(t)e^{A(t)}\dd{t}}, \quad \textrm{con} \quad A(t) = \int_{t_0}^t a_0(t)\dd{t}}
	\end{equation}
	Applicata al nostro caso $\dot x-ax=b\bar u$, sapendo che $A(t) = -at$, abbiamo:
	\begin{equation*}
		x(t) = e^{at}\sparen{10+b\bar u\int_0^t e^{-at} \dd{t}} = 10 e^{at}+ b\bar ue^{at}\sparen{-\frac{e^{-at}}{a}}_0^t = 10e^{at} +\frac{b\bar u}{a} e^{at}(-e^{-at}+ 1) = 10e^{at} + \frac{b\bar u}{a}(e^{at} - 1).
	\end{equation*}
\end{esem}

\begin{defin}{Equilibrio}{equilibrio}
	Dato un sistema dinamico \textit{non forzato}, vale che:
	\begin{equation}
		x_e\in \R^n \ \ \textrm{\textbf{equilibrio}} \ \leftrightarrow \ \forall t \geq t_0 \ \ x(t)=x_e \ \ \textrm{è traj.}
	\end{equation}
	In altre parole, si dice che $x_e$ è un equilibrio se la funzione costante associata $x(t) = x_e$ è una \textbf{traiettoria di stato}. In caso di \textit{sistemi invarianti continui}, data la costanza della traiettoria, si avrà  variazione di stato nulla:
	\begin{equation*}
		\dot x(t) = f(x(t)) \xrightarrow[eq.]{}\boxed{f(x_e) = 0}
	\end{equation*}
	Dunque, in corrispondenza di un equilibrio il sistema rimane piantato nello stato in cui si trova.
	\bb
	Nel caso più generale in cui abbiamo un sistema dinamico \textit{forzato}, vale che
	\begin{equation}
			(x_e, u_e)\in \R^n \times \R^m \ \ \textrm{\textbf{coppia di equilibrio}} \ \leftrightarrow \ \forall t \geq t_0 \ \ (x(t), u(t))=(x_e, u_e) \ \ \textrm{è traj.}
	\end{equation}
	e analogamente avremo (sempre per sistemi invarianti) $\boxed{f(x_e,u_e) = 0}$ per gli stessi motivi visti.
\end{defin}
Possiamo dire che $(x_e, u_e)$ è una coppia di equilibrio per un sistema se, applicando l'ingresso $u_e$ in presenza di stato $x_e$, il sistema non si schioda da quello stato.
\begin{lemma}
In caso di sistema in equilibrio, vale che \textbf{anche l'uscita $y(t)$ rimarrà costante ad un valore $y_e$, chiamato uscita di equilibrio.}
\end{lemma}
\begin{lemma}
	In caso di sistemi non invarianti, un equilibrio è tale quando annulla la $f$ per ogni t.
\end{lemma}

\begin{esem}
	Preso l'esercizio sul pendolo (appunti cartacei), calcolare l'equilibrio del sistema ad esso associato (che riportiamo qui a meno dell'equazione di uscita):
	\begin{equation*}
	x \in \R^2, \quad
		\begin{dcases}
			\dot x_1(t) = x_2(t) \\
			\dot x_2(t) = \frac{g}{l}\sin\theta -\frac{b}{Ml^2}x_2(t)+\frac{1}{Ml^2}u(t)
		\end{dcases}
	\end{equation*}
	Abbiamo uno stato in $\R^2$ e un'ingresso scalare, per cui la coppia di equilibrio che bisognerà trovare sarà
	\begin{equation*}
		(x_e, u_e) \quad \textrm{con} \quad x_e \in \R^2, \ u_e \in \R.
	\end{equation*}
	 Notiamo anche che il sistema è tempo invariante, in quanto non compare alcuna dipendenza dal tempo scorrelata da stato ed ingresso, per cui possiamo utilizzare la considerazione che \textbf{in presenza di un equilibrio vale:}
	\begin{equation*}
		f(x_e,u_e) = 0 \quad \rightarrow \quad  \begin{bmatrix}
			x_{e2} \\ \frac{g}{l}\sin\theta -\frac{b}{Ml^2}x_{e2}+\frac{1}{Ml^2}u_e
		\end{bmatrix} = \begin{bmatrix}
			0 \\ 0
		\end{bmatrix}
	\end{equation*}
	Eseguendo i calcoli otteniamo:
	\begin{equation*}
		\begin{dcases}
			x_{e2} = 0 \\
			u_e = - Mlg\sin\theta 
		\end{dcases}
	\end{equation*}
\end{esem}

\section{Classificazioni}
Prima di vedere i concetti matematici, raggruppiamo i sistemi che ci possono capitare secondo queste considerazioni (indichiamo prima la classe generale, poi una sua sottoclasse particolare):
\begin{itemize}
	\item \textbf{multivariabili (MIMO) o monovariabili (SISO)} \rarr nel secondo caso abbiamo una dimensione del vettore di ingressi e di quello delle uscite pari ad 1: $$ \dim{u(t)} = m = \dim{y(t)} = p = 1.$$ 
	\item \textbf{propri o strettamente propri} \rarr nei \textbf{primi} vale l'espressione generale della forma di stato, in particolare \textbf{l'uscita dipende direttamente dagli ingressi} (c'è ovviamente anche la dipendenza con lo stato ed, eventualmente, con il tempo $t$); nei \textbf{secondi} invece \textbf{l'uscita $y(t)$ non dipende direttamente dagli ingressi}, per cui la funzione $h$ sarà del tipo:
	\begin{align*}
		y(t) = h(x(t),t).
	\end{align*}
	Il vantaggio dei sistemi strettamente propri è che, per il calcolo dell'uscita è obbligatorio passare per l'equazione differenziale, e questa tipicamente si comporta come un \textit{filtro}. Questo non accade nei propri, in quanto vale $y(t) = h(x(t), \mathbf{u(t)}, t)$. Dunque, questa azione filtrante non c'è e, se per esempio ci fossero delle discontinuità in in, queste potrebbero apparire anche in out.
	
\resource{0.8}{propri}{Questo sistema è composto da una parte strettamente propria (cerchiata), in cui l'uscita non dipende direttamente dagli ingressi; presenta tuttavia una componente in basso che invece ha una dipendenza diretta. Di conseguenza, visto in toto, è proprio.}
\end{itemize}

\begin{itemize}
	\item \textbf{forzati o non forzati} \rarr questi ultimi \textbf{non ammettono ingressi, per cui evolvono per conto proprio}:
		\begin{equation*}
		\begin{dcases}
			\dot x(t)=f(x(t), t) \\
			y(t)=h(x(t),t)
		\end{dcases}
\end{equation*} 
	 Nei forzati, invece, gli ingressi sono significativi. È una definizione più stringente rispetto a quella vista per i sistemi propri o strettamente propri, in quanto in questo caso l'eventuale non dipendeza \textit{non riguarda solo l'equazione di uscita, ma anche quella di stato}:
A noi interessa progettare \textit{sistemi di controllo, capaci di modificare opportunamente gli ingressi per ottenere un certo comportamento.} È ovvio che quindi il focus verterà sui sistemi forzati.
\item\textbf{tempo varianti ed invarianti} \rarr in questi ultimi \textbf{le funzioni di stato e uscita $f, h$ non dipendono esplicitamente dal tempo $t$}, per cui si presenteranno in questa forma (questa cosa è dimostrabile):
\begin{equation*}
\begin{dcases}
		\dot x(t) = f(x(t), u(t)) \\
		y(t) =  h(x(t), u(t))
\end{dcases}
\end{equation*}
il che significa che tutti i parametri che non siano stato e ingressi all'interno delle funzioni suddette si comportano in modo \textit{costante.}
Inutitivamente, un sistema tempo invariante \textbf{si comporterà, a parità di ingressi, in modo identico indipendentemente da quando viene azionato}. In altre parole, se un ingresso $k(t)$ genera un'uscita $z(t)$, allora per ogni ingresso ritardato del tipo $k(t+\Delta)$ si otterrà un uscita $z(t+\Delta),$ ossia ritardata dello stesso quantitativo temporale.
\item \textbf{non lineari e lineari} \rarr in questi ultimi \textbf{entrambe le equazioni di stato ed uscita dipendono linearmente dallo stato e dall'ingresso, dunque sono scrivibili come combinazioni lineari degli elementi di $x(t)$ e di $u(t)$}. Simbolicamente (viene omessa la dipendenza dal tempo per brevità):
\begin{align}
	\label{f_lineare}
	f = \begin{bmatrix}
		f_1(x, u, t) \\
		\vdots \\
		f_n(x, u, t)
	\end{bmatrix} \quad \textrm{con} \quad f_j(x,u,t) & = a_{j1}(t)x_1(t) + \cdots + a_{jn}(t)x_n(t) + b_{j1}(t)u_1(t) + \cdots + b_{jm}(t)u_m(t)
\end{align}
\begin{align}
\label{h_lineare}
		h = \begin{bmatrix}
		h_1(x, u, t) \\
		\vdots \\
		h_p(x, u, t)
	\end{bmatrix} \quad \textrm{con} \quad h_k(x,u,t) & = c_{k1}(t)x_1(t) + \cdots + c_{kn}(t)x_n(t) + d_{k1}(t)u_1(t) + \cdots + d_{km}(t)u_m(t)
\end{align}
con $j \in \{1,2, \cdots, n\}$ e $k \in \{1,2, \cdots, p\}$.
\end{itemize} 


\section{Stabilità di uno stato di equilibrio e di una traiettoria}
\begin{prop}
Dato un sistema avente \textbf{ingresso fissato ad un valore $u_e$},
questo solo apparentemente è forzato: \textbf{non potendo cambiare l'ingresso}, è possibile \textbf{vederlo, almeno concettualmente, come un non forzato.} Nei prossimi concetti si supporrà di lavorare in questa situazione:
\begin{equation*}
	\dot x(t) = f(x(t), u_e) \quad \rightsquigarrow \quad  \dot x(t) = f(x(t)).
\end{equation*}
\end{prop}
Abbiamo visto che $(x_e, u_e)$ è una coppia di equilibrio per un sistema quando annullano la funzione $f$. In particolare, applicando un ingresso $u_e$ ad un sistema avente stato iniziale $x_e$, l'evoluzione del sistema è stazionaria.
\begin{equation*}
	(x_e, u_e) \quad  \textrm{equilibrio} \quad \rightarrow \quad \dot x(t) = f(x_e, u_e) = 0 \quad \textrm{cioè} \quad x(t) = x_e \ \forall t.
\end{equation*} Cosa succede se, però, invece di partire all'equilibrio, partissi da un punto $x(0)$ situato \textbf{vicino ad esso?} 
\begin{equation*}
	 \boxed{x(0)=x_e+\Delta x_0} \quad \rightarrow \quad  x(t) = \ ?
\end{equation*}


\begin{defin}{Equilibrio stabile ed instabile}{}
Uno stato di equilibrio $x_e$ si dice \textbf{stabile} se:
\begin{equation}
	\forall \epsilon > 0 \ \ \exists \delta > 0 \ \ : \ \ \forall x_0 \ \textrm{con} \ \norm{x_0 - x_e} \leq \delta \ \textrm{risulta} \ \norm{x(t) - x_e} \leq \epsilon \ \forall t.
\end{equation} 	
A parole: per ogni intorno di $x_e$ di raggio $\epsilon$ piccolo a piacere, è possibile prendere un intorno di raggio $\delta$ da $x_e$ e, se $x_e$ è stabile, \textbf{partendo da un qualsiasi punto $x_0$ che disti da $x_e$ meno di $\delta$} (i.e. prendendo qualunque punto $x_0$ all'interno dell'intorno di raggio $\delta$), \textbf{la traiettoria per tutti i tempi successivi rimane sempre nell'intorno di raggio $\epsilon$.}
\bb
Se invece $x_e$ è \textbf{instabile}, non vale quanto detto, i.e. possiamo prendere un intorno di raggio $\epsilon$ e, comunque si scelga un secondo intorno di raggio $\delta$, esiste un punto $x_0$ all'interno di quest'ultimo tale che, \textbf{la sua traiettoria per un qualche istante di tempo $t$} (al limite, per ogni $t$) \textbf{esce dall'intorno di raggio $\epsilon$.}
\begin{equation*}
	\exists \epsilon > 0 \ \ \forall \delta > 0 \ \ \textrm{in cui} \ \ \exists x_0 \ \textrm{con} \ \norm{x_0 - x_e} \leq \delta \ \ \textrm{tale per cui} \ \ \exists t \ : \ \norm{x(t) - x_e} > \epsilon.
\end{equation*} 
\end{defin}
\textbf{Osservazione:} Perché considerare due intorni uno dentro l'altro? Perché potrebbero esistere punti all'interno di $\epsilon$ la cui traiettoria nel tempo \textit{non rimane confinata in $\epsilon$ stesso!}
\begin{defin}{Equilibrio attrattivo}{}
Uno stato di equilibrio $x_e$ si dice \textbf{attrattivo} se:
\begin{equation}
	\exists \delta > 0 \ \ : \ \ \forall x_0 \ \textrm{con} \ \norm{x_0 - x_e} \leq \delta \ \textrm{risulta} \ \llimit{t}{\pinf}{\norm{x(t)-x_e}} = 0.
\end{equation}	
A parole: per ogni intorno di $x_e$ di raggio $\delta$ piccolo a piacere, è possibile prendere \textbf{un qualsiasi punto $x_0$ all'interno di questo intorno} e, se $x_e$ è attrattivo, \textbf{all'infinito la traiettoria $x(t)$ convergerà verso $x_e$ stesso.} 
\bb
Attenzione! Questa definizione non dice niente sul comportamento della traiettoria nel corso del tempo: potrà anche allontanarsi di 3000km da $x_e$; se all'infinito convergerà verso quest'ultimo, sarà attrattivo (in altre parole, non c'è un intorno di raggio $\epsilon$ all'interno del quale la traiettoria deve rimanere sempre confinata).
\end{defin}
Dunque, partendo da un punto $x(0)$ vicino all'equilibrio $x_e$ si hanno comportamenti diversi: se la sua traiettoria $x(t)$, per tutti i tempi successivi, rimane confinata entro un intorno vicino ad $x_e$, si ha la \textbf{stabilità}; se invece per $t \rightarrow \pinf$ andrà a convergere verso $x_e$ (nel frattempo potrà fare quello che vuole) si ha \textbf{attrattività}. 
\bb
\begin{defin}{Equilibrio asintoticamente stabile}{}
	Uno stato di equilibrio $x_e$ si dice \textbf{asintoticamente stabile} se è \textbf{stabile e attrattivo}. Questo significa che, preso un intorno di raggio $\epsilon$ da $x_e$, un \textbf{qualsiasi punto} $x(0)$ all'interno di un intorno di raggio $\delta$ da $x_e$, godrà di una traiettoria $x(t)$ che resterà \textbf{confinata per ogni $t$ entro l'intorno di raggio $\epsilon$}, e che \textbf{per $t \rightarrow \pinf$} convergerà verso $x_e$. In simboli:
	\begin{equation*}
		\forall \epsilon > 0 \ \ \exists \delta > 0 \ \ : \ \ \forall x_0 \ \textrm{con} \ \norm{x_0 - x_e} \leq \delta \ \textrm{risulta} \ \norm{x(t) - x_e} \leq \epsilon \ \forall t \ \textrm{e} \llimit{t}{\pinf}{\norm{x(t)-x_e}} = 0.
	\end{equation*}
	Questa situazione ci piace particolarmente! L'obiettivo sarà infatti progettare sistemi di controllo che generino questo tipo di comportamento, i.e. convergente verso, appunto, una stabilità.
\end{defin}
\bb
\textbf{Osservazione:} Nelle definizioni di stabilità (asintotica o meno), abbiamo dato per scontato che l'intorno di raggio $\delta$ sia contenuto all'interno di quello di raggio $\epsilon$. Questo ha senso: Se l'intorno di raggio $\delta$ fosse più grande, potremmo prendere un punto al suo interno che, ancor prima di partire, si trova già al di fuori dall'intorno di raggio $\epsilon$. La proprietà di stabilità sarebbe già falsa. 
\bb
\textbf{Osservazione:} Quelle definite finora sono stabilità \textit{locali}, in quanto hanno valore vicino all'equilibrio $x_e$. È possibile ottenere le rispettive \textbf{definizioni globali rimuovendo i vincoli dipendenti da $\delta$ (dominio di attrazione), e quindi supponendo $x$ libero di variare nel suo dominio di definizione, che nel caso generale è $\R^n.$}
\bb
Quanto visto vale anche per una intera \textit{traiettoria}, non solo per un singolo punto $x_e$. Vediamo per brevità soltanto la definizione di \textit{asintoticamente stabile} (che comprende le altre).
\begin{defin}{Traiettoria asintoticamente stabile}{}
Una traiettoria per un sistema dinamico $\bar x(t), \ t \geq t_0$ si dice \textbf{asintoticamente stabile} se è \textbf{stabile ed attrattiva}, ossia, in simboli:
\begin{itemize}
	\item \textbf{stabilità} \rarr \ $\forall \epsilon > 0 \ \  \exists \delta > 0 \ \ : \ \ \forall t_\delta \ \textrm{con} \ \norm{t_\delta - t_0} \leq \delta \ \textrm{risulta} \ \norm{x(t)_\delta -\bar x(t)} \leq \epsilon$; 
	\item \textbf{attrattiva} \rarr $\exists \delta > 0 \ : \ \forall t_\delta \ \textrm{con} \ \norm{t_\delta - t_0} \leq \delta \ \textrm{risulta} \ \displaystyle \llimit{t}{\pinf}{\norm{x(t)_\delta -\bar x(t)}} = 0$.
\end{itemize}
A parole, vale la proprietà per $\bar x(t)$ se, scelto un intorno di raggio $\epsilon$ della traiettoria e un intorno $\delta$ della condizione iniziale $t_0$, partendo da un generico punto $t_\delta$ all'interno dell'intorno $\delta$ \textbf{la traiettoria che ne deriverà resterà confinata per ogni $t$ entro l'intorno di raggio $\epsilon$ della traiettoria di partenza e, per $t \rightarrow \pinf$, andrà a convergere verso $\bar x(t)$.}
\end{defin}

\chapter{Sistemi dinamici lineari}
Vediamo ora un caso particolare di sistemi dinamici: è chiaro che più particolarizziamo, più si va incontro ad una semplificazione dei modelli matematici necessari per la descrizione del sistema. Questa semplificazione è direttamente proporzionale alla semplicità di studio. Più uno studio è semplice, più cose riusciremo a dire al riguardo.
\bb
Un grande vantaggio di cui godono i sistemi lineari è quello di poter essere descritti completamente in \textit{forma matriciale.} Infatti, alla luce di quanto visto in \eqref{f_h_matrices}, e riprendendo le scritture in forma di combinazioni lineari \eqref{f_lineare} e \eqref{h_lineare}, definiamo la

\begin{defin}{Scrittura in forma matriciale di un sistema lineare}{}
	Vediamo l'equazione di stato:
	\begin{align}
		\begin{cases}\underbrace{\begin{bmatrix}
			\dot x_1(t) \\
			\vdots \\
			\dot x_n(t)
		\end{bmatrix}}_{(n\times 1)} = 
		\underbrace{\begin{bmatrix}
			a_{11}(t) & a_{12}(t) & \cdots &a_{1n}(t) \\
			\vdots & \vdots & \ddots & \vdots \\
			a_{n1}(t) & a_{n2}(t) & \cdots &a_{nn}(t)
		\end{bmatrix} \begin{bmatrix}
			x_1(t) \\ \vdots \\ x_n(t)
		\end{bmatrix}}_{(n \times n)(n\times 1) \rightarrow (n\times 1)} + \underbrace{\begin{bmatrix}
			b_{11}(t) & b_{12}(t) & \cdots &b_{1m}(t) \\
			\vdots & \vdots & \ddots & \vdots \\
			b_{n1}(t) & b_{n2}(t) & \cdots &b_{nm}(t)
		\end{bmatrix} \begin{bmatrix}
			u_1(t) \\ \vdots \\ u_m(t)
		\end{bmatrix}}_{(n \times m)(m\times 1) \rightarrow (n\times 1)}
\\
\\
		\underbrace{
			\begin{bmatrix}
			y_1(t) \\ \vdots \\ y_p(t)
			\end{bmatrix}}_{(p\times 1)} = 
		\underbrace{\begin{bmatrix}
			c_{11}(t) & c_{12}(t) & \cdots &c_{1n}(t) \\
			\vdots & \vdots & \ddots & \vdots \\
			c_{p1}(t) & c_{p2}(t) & \cdots &c_{pn}(t)
		\end{bmatrix} \begin{bmatrix}
			x_1(t) \\ \vdots \\ x_n(t)
		\end{bmatrix}}_{(p \times n)(n\times 1) \rightarrow (p\times 1)} + \underbrace{\begin{bmatrix}
			d_{11}(t) & d_{12}(t) & \cdots &d_{1m}(t) \\
			\vdots & \vdots & \ddots & \vdots \\
			d_{p1}(t) & d_{p2}(t) & \cdots &d_{pm}(t)
		\end{bmatrix} \begin{bmatrix}
			u_1(t) \\ \vdots \\ u_m(t)
		\end{bmatrix}}_{(p \times m)(m\times 1) \rightarrow (p\times 1)}
		\end{cases}
\end{align}
	Assegnando dei nomi a queste matrici, otteniamo una scrittura elegante della \textbf{forma di stato per un sistema lineare tempo variante}:
	\begin{equation}
		\begin{cases}
			\dot x(t) = A(t)x(t) + B(t)u(t) \\
			y(t) = C(t)x(t) + D(t)u(t)
		\end{cases}
	\end{equation}
	con $A(t) \in \mathcal{M}(n,n,\R)$, $B(t) \in \mathcal{M}(n,m,\R)$, $C(t) \in \mathcal{M}(p,n,\R)$, $D(t) \in \mathcal{M}(p,m,\R)$.
Dunque, se il modello che traggo dall'analisi di un sistema è lineare, posso \textbf{descriverlo completamente mediante le quattro matrici} descritte sopra ($A,B$ per lo stato; $C,D$ per l'uscita).
\end{defin}

\begin{defin}{LTI - Sistemi lineari \textit{tempo invarianti}}{}
	La definizione vista sopra si semplifica ulteriormente nel caso in cui il SL sia \textbf{tempo invariante} (\textbf{sistema LTI}), in quanto in quel caso solo stato e ingressi presenteranno una dipendenza col tempo, col risultato che $A,B,C,D$ diventeranno matrici \textit{di coefficienti:}
\begin{equation}
\label{eq:lti}
		\begin{cases}
			\dot x(t) = Ax(t) + Bu(t) \\
			y(t) = Cx(t) + Du(t)
		\end{cases}
\end{equation}
Le dimensioni delle matrici restano tali.
\end{defin}
\begin{defin}{LTI SISO}{}
Molto comuni in questo corso saranno gli \textbf{LTI SISO} (dunque con $m=p=1$). L'equazione sarà analoga a quella vista per gli LTI generici \eqref{eq:lti}; cambierà però la dimensione delle varie matrici:
\begin{align*}
\begin{cases}
\underbrace{\begin{bmatrix}
			\dot x_1(t) \\
			\vdots \\
			\dot x_n(t)
		\end{bmatrix}}_{(n\times 1)} = 
		\underbrace{\begin{bmatrix}
			a_{11} & a_{12} & \cdots &a_{1n} \\
			\vdots & \vdots & \ddots & \vdots \\
			a_{n1} & a_{n2} & \cdots &a_{nn}
		\end{bmatrix} \begin{bmatrix}
			x_1(t) \\ \vdots \\ x_n(t)
		\end{bmatrix}}_{(n \times n)(n\times 1) \rightarrow (n\times 1)} + \underbrace{\begin{bmatrix}
			b_{11} \\ \vdots \\ b_{n1} 
		\end{bmatrix} u(t)}_{(n \times 1)(1\times 1) \rightarrow (n\times 1)}
\\
\\
		\underbrace{
			y(t)}_{(1\times 1)} = 
		\underbrace{\begin{bmatrix}
			c_{11} & c_{12} & \cdots &c_{1n}
		\end{bmatrix} \begin{bmatrix}
			x_1(t) \\ \vdots \\ x_n(t)
		\end{bmatrix}}_{(1 \times n)(n\times 1) \rightarrow (1\times 1)} + \underbrace{
			d_{11} u(t)}_{(1 \times 1)(1\times 1) \rightarrow (1\times 1)}
\end{cases}
\end{align*}
con $A \in \mathcal{M}(n,n, \R)$, $B \in \mathcal{M}(n,1, \R)$, $C \in \mathcal{M}(1,n, \R)$, $D \in \mathcal{M}(1,1, \R)$.
\end{defin}



\section{Principio di sovrapposizione degli effetti}
Attenzione! Vediamo ora questa proprietà riferita alla rappresentazione di un sistema dinamico lineare mediante\textit{ forma di stato}, ma questa continuerà a valere anche per altri modelli matematici che vedremo successivamente.
\bb
\begin{defin}{}{}
\textbf{Attenzione! Valida solo per sistemi lineari, sia tempo varianti che invarianti.}
	\bb
	Consideriamo il sistema lineare seguente
	\begin{equation*}
	\begin{cases}
		\dot x(t) = A(t)x(t) + B(t)u(t) \\
		y(t) = C(t)x(t) +D(t)u(t)
	\end{cases}
	\end{equation*}
	e supponiamo di avere \textbf{due traiettorie} per esso (i.e. evoluzioni stato-ingresso che soddisfano l'equazione di stato):
	\begin{equation*}
		(x_a(t), u_a(t))\quad traj \ con \quad  x_a(t_0)=x_{0a},
	\end{equation*}
		\begin{equation*}
		(x_b(t), u_b(t))\quad traj \ con \quad x_b(t_0)=x_{0b}.
	\end{equation*}
Allora, \textbf{la combinazione lineare delle due genera una nuova traiettoria} per il sistema, ossia una nuova soluzione per l'equazione di stato. $\forall \alpha, \beta \in \R$
	\begin{equation}
		(x_{ab}(t), u_{ab}(t)) = (\alpha x_a(t)+\beta x_b(t), \alpha u_a(t) + \beta u_b(t)) \quad traj \ con \quad x_{ab}(t_0) = \alpha x_{0a} + \beta x_{0b}.
	\end{equation}
\end{defin}

\begin{esem} Prendiamo un LTI avente stato bidimensionale e condizioni iniziali $x_1(0) = x_{1,0}$ e $x_2(0) = x_{2,0}$:
	\begin{equation*}
		\begin{cases}
			\dot x_1(t)=x_2(t) \\
			\dot x_2(t) = u(t)
		\end{cases} \quad \rightarrow \quad  \dot x(t) = \begin{bmatrix}
			0 & 1 \\ 0 & 0
		\end{bmatrix}\begin{bmatrix}
		x_1(t) \\ x_2(t) \end{bmatrix}+\begin{bmatrix}
			0 \\ 1
		\end{bmatrix} u(t) = Ax(t)+Bu(t).
	\end{equation*}
\subsubsection{Traiettoria A}
Consideriamo $u_a (t) = 3$, $t \geq 0$, e una condizione iniziale per lo stato del tipo $x(0) = \begin{bmatrix}
	2 \\ 0
\end{bmatrix}$. Integrando da $0 $ a $t$ ciascuna componente dello stato, troviamo:
\begin{equation*}
	x_2(t)- x_2(0) = \int_0^t u(t)\dd{t} \ \rightarrow \ x_2(t) = 3t
\end{equation*}
\begin{equation*}
	x_1(t) -x_1(0) = \int_0^t x_2(t)\dd{t} \ \rightarrow \ x_1(t) = 2+\frac{3}{2}t^2
\end{equation*}
Definiamo dunque:
\begin{equation*}
\boxed{
	Traj \ A =(x_A(t), u_A(t)) = \paren{\begin{bmatrix}
		2+\frac{3}{2}t^2 \\ 3t
	\end{bmatrix}, 3}}
\end{equation*}
È una traiettoria perché se la infiliamo all'interno del sistema definito all'inizio, lo soddisfa.

\subsubsection{Traiettoria B}
Consideriamo $u_b(t) = \cos(t)$, $t \geq 0$, e una condizione iniziale per lo stato $x(0) = \begin{bmatrix}
	3 \\ 1
\end{bmatrix}$. Integrando:
\begin{equation*}
	x_2(t) = 1+\sin(t) \quad \quad x_1(t) =3è+\int_0^t1+\sin(t)\dd{t} = 3
	+t+[-\cos(t)]^t_0 = 3+t-\cos(t) + 1 = 4+t-\cos(t).
\end{equation*}
Definiamo dunque:
\begin{equation*}
	\boxed{Traj \ B = (x_b(t), u_b(t)) = \paren{\begin{bmatrix}
		4+t-\cos(t) \\ 1+\sin(t)
	\end{bmatrix}, \cos(t)}}
\end{equation*}

\subsubsection{La combinazione delle traiettorie A, B sarà ancora una traiettoria?}
Consideriamo  $x_{ab} = \alpha x_a + \beta x_b$ e $u_{ab} = \alpha u_a + \beta u_b$.  Nel nostro caso, avremo:
\begin{equation*}
	Traj \ C = (x_{ab}(t), u_{ab}(t)) = \paren{\begin{bmatrix}
		\alpha(2+\frac{3}{2} t^2) + \beta(4+t-\cos(t)) \\ 3\alpha t + \beta(1+\sin(t))
	\end{bmatrix}, 3\alpha+\beta \cos(t) }
\end{equation*}
Questa è una traiettoria? Dovrebbe risolvere l'equazione di stato $\forall \alpha, \beta \in \R$:
\begin{align*}
	\dot x_2(t) = 3\alpha +\beta \cos (t) = u(t) \rightarrow yes \quad \quad \dot x_1(t) = 3\alpha t +\beta +\beta \sin(t) = x_2(t) \rightarrow yes
\end{align*}
E infatti la risolve. Dunque la sovrapposizione funziona.
\end{esem}
\textbf{Attenzione!} Quanto visto è stato applicato solamente all'equazione di stato (e quindi considerando la \textit{traiettoria di stato}), ma vale in modo analogo anche per la \textit{traiettoria di uscita}, ossia della funzione del tipo $(x(t), u(t))$ che soddisfa $y(t)$.

\section{Traiettoria come somma di due evoluzioni}
\begin{defin}{}{}
Utilizzando il principio di sovrapposizione, vale che \textbf{fissato uno stato inziale $x(t_0) = x_0$ e applicando un ingresso $u(t)$, $t \geq t_0$, è possibile scrivere la traiettoria di stato di un sistema lineare (tempo variante od invariante) come somma di due traiettorie:}
\begin{equation}
	x(t) = x_L(t) +x_F(t)
\end{equation}
dove:
\begin{itemize}
	\item  $x_L(t)$, $t \geq t_0$ è la traj. di stato ottenuta \textbf{partendo con uno stato iniziale diverso da zero ed ingresso nullo}, i.e. $x_L(t_0) = x_0$ e $u_L(t) = 0 \ \forall t \geq t_0$ \rarr \textbf{evoluzione libera};
	\item $x_F(t)$, $t \geq t_0$ è la traj di stato ottenuta \textbf{partendo con uno stato iniziale nullo ed un ingresso diverso da zero}, i.e. $x_F(t_0) = 0$ e $u_F(t) = u(t) \ \forall t \geq t_0$ \rarr \textbf{evoluzione libera.}
\end{itemize}
\end{defin}

\begin{prop}
\label{prop:sl_coppia_nulla_eq}
	Consideriamo un sistema lineare, quindi definito da un'equazione di stato del tipo $\dot x(t) = A(t)x(t) + B(t)u(t).$ Supponendo ingresso nullo (dunque $u(t) = 0$)  e stato nullo ($x(t)=0$), allora si vede che l'intera ED si annulla: $\dot x(t) = f(x(t), u(t), t) = 0 \quad \forall t$. Questa è la definizione di \textbf{equilibrio.} In particolare, vale che \textbf{per un sistema lineare la coppia  $(x_e,u_e) = (0,0)$} \textbf{è sempre di equilibrio}, ed è un caso particolare  in cui il sistema non fa niente per sempre.
\end{prop}

\begin{prop}
	Guardando i risultati precedenti, abbiamo scoperto che è possibile ottenere una generica traettoria per un SL \textbf{perturbando l'equilibrio prima mediante cambiamento di stato iniziale - tenendo fermo l'ingresso, evoluzione libera - e poi tenendo nullo lo stato iniziale -  tenendolo all'equilibrio - e applicandogli un ingresso.} Quanto visto vale inoltre \textbf{anche per l'equazione dell'uscita:}
	\begin{equation}
		y(t) = y_L(t) + y_F(t).
	\end{equation}
\end{prop}

\section{Sistemi dinamici lineari tempo invarianti (LTI)}

\subsection{Equazione della traiettoria nel caso scalare e generale}
Se supponiamo di avere un sistema LTI \textit{scalare}, i.e. con $x(t),y(t),u(t) \in \R$, possiamo \textbf{esprimere esplicitamente le equazioni di traiettoria, sia dello stato, che dell'uscita}, mediante la risoluzione dell'equazione differenziale dello stato tramite la formula riquadrata vista qualche pagina fa:
\begin{align*}
	\begin{dcases}
	\dot x(t) = ax(t) + bu(t) \\
	x(0) = x_0
	\end{dcases} \ \rightarrow \ \boxed{x(t) =e^{at}x_0+\underbrace{\int_0^te^{a(t-\tau)}bu(\tau)\dd{\tau}}_{int. \ di \ convoluzione}}
\end{align*}
Lo stato inizia da $t= 0$ quindi l'estremo inferiore dell'integrale è $t_0  = 0.$ Otteniamo poi l'espressione dell'uscita semplicemente sostituendo l'equazione riquadrata al posto di $x(t)$:
\begin{align*}
	y(t) = cx(t) +du(t) \ \rightarrow \ \boxed{y(t) = ce^{at}x_0+c\int_0^te^{a(t-\tau)}bu(\tau)\dd{\tau} + du(t)}
\end{align*}
\bb
È possibile estendere queste equazioni al \textbf{caso generale, i.e. un sistema LTI con $x \in \R^n$, $y \in \R^p, u \in \R^m$?} Sì, a patto che definiamo cosa vuol dire \textit{calcolare l'esponenziale di una matrice}. Questo perché quello che nel caso precedente era $a$, adesso è una matrice!
\begin{equation*}
	e^{at} = 1+at + \frac{(at)^2}{2!} + \cdots  \quad \longrightarrow \quad e^{At} = I+ At+\frac{(At)^2}{2!} + \cdots 
\end{equation*}
Fortunatamente, l'elevamento a potenza è un'operazione inclusa nello spazio delle matrici \textbf{quadrate} (e infatti la matrice A è sempre quadrata $n\times n$), in quanto è semplicemente un prodotto. Posso dunque definire l'equazione di una generica traiettoria anche nel caso generale:
\begin{align*}
	\boxed{x(t) =e^{At}x_0+\int_0^te^{A(t-\tau)}Bu(\tau)\dd{\tau} \quad \quad y(t) = Ce^{At}x_0+C\int_0^te^{A(t-\tau)}Bu(\tau)\dd{\tau}  + Du(t)}	
\end{align*}
Notiamo che anche da un punto di vista dimensionale le cose tornano: $e^{At}$ è una somma infinita di matrici $n \times n$, dunque se questa serie converge il risultato è ancora una $n \times n$; $x_0$ è un vettore $n \times 1$ (perché è una condizione iniziale dello stato, che è in $\R^n$). Il loro prodotto è quindi $n \times 1$, compatibile con $x(t)$. Si dimostra che anche l'integrale di convoluzione è coerente.
\bb
\textbf{Nota!} Questa definizione di traiettoria rispecchia alla perfezione quanto è stato detto sulla traiettoria sistemi lineari, scrivibile come somma di due componenti. Guardando $x(t)$, vediamo che il primo termine, indipendente dall'ingresso, rappresenta l'\textit{evoluzione libera}; l'integrale di convoluzione quella \textit{forzata}. Questo è \textbf{potentissimo! L'evoluzione libera di un sistema LTI non scalare dipende solo ed unicamente dall'esponenziale di una matrice}. Se riuscissimo a capire il \textbf{comportamento di questo esponenziale} potremmo capire \textbf{l'evoluzione libera di un LTI forzato, o l'evoluzione in toto di un sistema lineare non forzato!}
\subsubsection{Cenni sulla rappresentazione in forma di Jordan di una matrice}
Calcolare l'esponenziale di una matrice definito come somma di infiniti termini non è però fattibile. Fortunatamente, mediante la \textbf{rappresentazione di Jordan} delle matrici, questo calcolo si traduce in un concetto molto più semplice, e soprattutto, finito. Diamo qualche risultato preliminare.
Partiamo con delle definizioni preliminari:
\begin{prop}
	Sia $M \in \mathcal{M}(n,n,\mathbb{K})$ e sia $\lambda_0$ un suo autovalore. Si chiama:
	\begin{itemize}
		\item \textbf{molteplicità algebrica} di $\lambda_0$  \rarr \textbf{il numero che esprime quante volte $\lambda_0$ annulla il polinomio caratteristico} associato ad $M$. La somma delle molteplicità algebriche \textbf{non può mai superare l'ordine della matrice};
		\item \textbf{molteplicità geometrica} di $\lambda_0$ \rarr la \textbf{dimensione dell'autospazio generato da $\lambda_0$, calcolato come $n - \rank{(M-\lambda_0 I)}$.} Le nozioni sul rango non vengono riprese qui.
	\end{itemize}
	Sottolineiamo inoltre la relazione
\begin{equation}
		1 \leq m_g(\lambda_0) \leq m_a(\lambda_0)\leq n.
	\end{equation} 
\end{prop}


\begin{defin}{Forma canonica di Jordan}{}
Sia $M \in \mathcal{M}(n,n,\mathbb{K}).$ Se ad $M$ sono associati \textbf{autovalori tutti esistenti in $\mathbb{K}$}, allora è possibile rappresentarla in un'altra matrice $J$, detta in forma di Jordan, mediante il cambiamento di base:
\begin{align*}
	M = TJT^{-1}.
\end{align*}
$J$ è una matrice \textbf{diagonale a blocchi} strutturata in questo modo:
	\begin{itemize}
		\item lungo la diagonale principale sono presenti \textbf{tutti gli autovalori di $M$ contati con la loro molteplicità algebrica};
		\item tutti gli autovalori uguali tra di loro consecutivi sulla diagonale formano un \textbf{blocco di Jordan}. Avremmo quindi \textbf{un numero di blocchi sulla diagonale pari al numero di autovalori distinti;}
		\item Ciascun blocco di Jordan è composto da un \textbf{numero di box di Jordan pari alla molteplicità geometrica di quell'autovalore}. Se un blocco è di dimensioni maggiori di $1\times 1$, allora è necessario aggiungere degli 1 sopra la diagonale principale (solamente di quello specifico blocco). Se la molteplicità algebrica e geometrica non ci consente di prendere in modo univoco i box di Jordan, è necessario procedere con il calcolo di dimensioni elevate ad esponenti incrementali, fin quando non si è in condizioni non ambigue in termini di scelta delle box.\footnote{i.e. $m.g. = \dim({\ker{(M-\lambda_0 I)^k})},$ che lega la molteplicità geometrica al calcolo esplicito di una dimensione.}
	\end{itemize}
\end{defin}
Non andiamo oltre su questo argomento, in quanto le formule della pagina precedente non saranno mai utilizzate, in questo corso, in forma analitica (ci aiuterà MATLAB a risolverle), ma soprattutto perché \textbf{ci interessa relativamente quale sia l'equazione specifica della traiettoria di un sistema LTI: quello che ci interessa è come si comporta - all'incirca - al trascorrere del tempo.} Vediamo ora come fare per scoprirlo senza passare per tutti questi concetti non affatto banali.
\subsection{Traiettoria come combinazione di modi naturali}
Il comportamento del sistema nel corso del tempo lo si può approssimare grazie ai modi naturali e ai risultati che vedremo  ora:
\bb\begin{defin}{}{}
Prendiamo un sistema LTI generico:
\begin{equation*}
	\begin{dcases}
		\dot x(t) = Ax(t)+Bu(t) \\
		y(t) = Cx(t)+Du(t)
	\end{dcases} \quad x(0)=x_0 
\end{equation*}
Prendiamo la matrice $A$: siano $\lambda_1,\ldots, \lambda_r$, con $r \leq n$ autovalori reali o complessi coniugati ad essa associata, aventi molteplicità algebrica $m_1, \ldots, m_r \geq 0$ tali che la loro somma dia $n$ (somma non deve superare l'ordine della matrice, come detto sopra).
Vale che \textbf{le  componenti dell'evoluzione libera dello stato o dell'uscita del sistema lineare sono combinazioni lineari dei modi naturali}, ossia di funzioni del tipo:
\begin{equation}
	\boxed{t^{q-1}e^{\lambda_i t}} \quad \lambda_i \in \R \ oppure \ \mathbb{C} \ i.e. \ \lambda_i = \sigma_i + j\omega_i
\end{equation} Inoltre, \textbf{poiché l'uscita è lineare nello stato, anche l'evoluzione libera dell'uscita è esprimibile come combinazione lineare di modi naturali.} Potrebbero successivamente aggiungersi formule per l'approssimazione dell'evoluzione \textit{forzata}.
\end{defin}

\begin{prop}
	Se la matrice $A$ è reale e ammette un autovalore \textbf{complesso} $\lambda_i = \sigma_i + j\omega_i$, allora \textbf{anche il suo coniugato $\bar \lambda_i = \sigma_i - j\omega_i$ è autovalore di $A$}. Si dimostra inoltre che anche i coefficienti $\gamma_{jiq}$ associati a $\lambda, \bar \lambda$ sono a loro volta complessi coniugati.
\end{prop}
Nell'equazione della risposta forzata, i termini complessi coniugati sommati fra loro daranno luogo \textbf{a termini reali} del tipo $e^{\sigma_i t}\cos(\omega_i t+\phi_i)$, con opportuni valori di fase.
Ricordiamo infatti che:
\begin{equation*}
	e^{\lambda_i t} = e^{\sigma_i t}e^{j\omega_i t} = e^{\sigma_i t} \paren{\cos(\omega_i t ) + j\sin(\omega_i t)}
\end{equation*}

\subsection{Tipologie di modi naturali}
L'equazione dei modi naturali dipende dalla tipologia degli autovalori della matrice $A$:
\bb
$m.a. > m.g \rightarrow \left\{
\begin{tabular}{p{.7\textwidth}}
\begin{itemize}
	\item \textbf{reali} \rarr $\ t^{q-1}e^{\lambda_i t}$
	\item \textbf{complessi coniugati} \rarr $ \ t^{q-1}e^{\sigma_it}\cos(\omega_i t + \phi_i)$
\end{itemize}
\end{tabular}
\right.$
\bb
$m.a. = m.g \rightarrow \left\{
\begin{tabular}{p{.7\textwidth}}
\begin{itemize}
	\item \textbf{reali} \rarr $e^{\lambda_i t}$
	\item \textbf{complessi coniugati} \rarr $ \ e^{\sigma_i t}\cos(\omega_i t + \phi_i)$
\end{itemize}
\end{tabular}
\right.$ 
\bb
Notiamo da questa scrittura che \textbf{le uniche componenti capaci di modificare il comportamento dei modi in modo significativo sono sempre e solo entità reali.} In particolare, sarà interessante vedere cosa succede al variare di $\lambda_i$ nel caso di autovalori reali (sia semplici che non), e al variare di $\sigma_i$ nel caso di autovalori complessi coniugati (ancora, sia semplici che non).
\resource{0.4}{modi_nat_semplici}{\centering Variazione della traiettoria del modo in caso di autovalori \textbf{semplici} $(m.a. = m.g.)$, \\ dunque a sinistra $e^{\lambda_i t}$ e a destra $e^{\sigma_i t} \cos (\omega_i t + \phi_i)$.}

\resource{0.4}{modi_nat_nonsemp}{\centering Traiettoria del modo in caso di autovalori \textbf{non semplici} $(m.a. > m.g.)$, \\
dunque a sinistra $t^{q-1}e^{\lambda_i t}$ e a destra $t^{q-1}e^{\sigma_i t}\cos(\omega_i t + \phi_i)$.}

\bb
Si ha \textbf{convergenza del modo, e quindi dell'evoluzione libera, quando la parte reale dell'autovalore è negativa}. Il comportamento in caso di autovalore nullo è l'unico che varia: se l'autovalore è semplice, il sistema appare stabile (perché l'evoluzione è costante), ma resta una condizione da evitare. In caso di non semplicità, autovalori nulli non sono più considerabili validi, in quanto il termine polinomiale li rende divergenti. Ricordiamo: la divergenza indica eventuale \textit{rottura del sistema!} 
\bb
Graficando gli autovalori su un piano di Gauss, abbiamo che: a sinistra dell'asse immaginario, si hanno modi convergenti; a destra modi divergenti, a prescindere dalle considerazioni sulle molteplicità. Ciascun autovalore complesso ha anche il suo coniugato disegnato mediante simmetria rispetto all'asse reale. Cambia il comportamento sull'asse immaginario (quindi per autovalori a parte reale nulla): nel caso di autovalori semplici, si hanno modi costanti; nell'altro caso, divergenza.

\subsection{Stabilità di un sistema}
\begin{prop}
Abbiamo visto nella Proposizione \eqref{prop:sl_coppia_nulla_eq} che per un sistema \textbf{lineare} forzato la coppia $(0,0)$ è sempre di equilibrio. Per un non forzato, basta che sia $x_e = 0$ (in quanto $u_e = 0$ per definizione di non forzatura).
\end{prop}
\begin{prop}
	Si dimostra che nei i sistemi lineari \textbf{le proprietà di stabilità valgono per tutti gli equilibri e le traiettorie} del sistema. Di conseguenza, \textbf{studiare la stabilità in un punto equivale a studiarla in tutti gli altri.} Con questo concetto si definisce la \textbf{stabilità di un sistema.}
\end{prop}
\begin{proof}
	Supponiamo che $x_e$ sia un punto di equilibrio per il sistema LTI non forzato $\dot x(t) = Ax(t).$ Dalla definizione di equilibrio, allora, $f(x_e) = Ax_e = 0$. Studiamo ora il sistema in un generico punto $\tilde x(t)$, definito come la traiettoria calcolata al tempo $t$ rispetto ad $x_e$ (e non rispetto allo zero): 
\begin{equation*}
		\tilde x(t) = x(t) - x_e.
	\end{equation*}
	Sostituendo nell'equazione del sistema, abbiamo:
\begin{align*}
	\dv{t}(\tilde x(t)) = \dv{t}(x(t)-x_e) \longrightarrow \dot x(t) = Ax(t) & =A(\tilde x(t) + x_e)  = A\tilde x(t)+ \cancel{Ax_e} \\ & = A\tilde x(t),
	\end{align*}
	da cui il legame tra stato e un qualsiasi punto $\tilde x(t)$:
\begin{equation*}
	\dot x(t) = Ax(t) = A \tilde x(t) \longrightarrow x(t) = \tilde x(t).
\end{equation*}
\end{proof}
\begin{prop}
Preso un LTI, vale che se $$\ker (A) = \{ x \in \mathbb{K}^n \ : \ Ax = 0 \}= 0 \quad \rightarrow \quad x=0 \quad \textrm{è l'unico punto di equilibrio}$$ con $\mathbb{K}$ il campo degli elementi della matrice $A$. Questo è visibile dal fatto che la definizione del kernel non è altro che la funzione $f$ di un sistema LTI non forzato. Essendo un sottospazio vettoriale, il nullo è sempre un vettore dell'insieme (in questo caso, l'unico).\end{prop}

L'idea è quindi quella di \textbf{studiare l'origine} di un LTI, in quanto abbiamo visto essere sempre un punto di equilibrio: se questo è instabile, allora lo saranno tutti gli altri per cui si parla di \textbf{sistema instabile}. Questo è potentissimo: generalizzo un concetto studiando solo un caso particolare.
\begin{defin}{Teorema di Ljapunov}{}
	Consideriamo un sistema LTI (non c'è dipendenza dal tempo delle matrici). Questo è:
	\begin{itemize}
		\item \textbf{asintoticamente stabile} $\leftrightarrow$ \textbf{tutti gli autovalori della matrice $A$ hanno parte reale (strettamente) negativa.} In questo modo, infatti, lo stato sarà combinazione lineare di modi naturali convergenti. in simboli: \boxed{\forall \lambda_i, \ \ \Re{\lambda_i} < 0}
		\item semplicemente \textbf{stabile} $\leftrightarrow$ \textbf{tutti gli autovalori hanno parte reale minore o uguale a zero, ma gli autovalori a parte reale nulla sono necessariamente semplici}. Se questo non valesse avrei il termine polinomiale $t^{q-1}$ che  genererebbe un comportamento divergente. In simboli: \boxed{\forall \lambda_i \ : \ \Re{\lambda_i}= 0 \ \textrm{deve valere} \ m.a.(\lambda_i) = m.g.(\lambda_i)} 
	\end{itemize}
\end{defin}

\subsection{Retroazione statica dallo stato}
Cominciamo ad entrare nel vivo della materia. Quanto visto a livello puramente matematico è utilizzabile nella pratica per controllare un sistema dinamico (ad esempio, renderlo asintoticamente stabile se non lo è). Vediamo meglio. Un motore elettrico è un immediato esempio di sistema LTI.
\bb
Prendiamo l'equazione di stato un sistema LTI forzato:
 $$\dot x(t) = Ax(t) + Bu(t),$$ e supponiamo di avere $x_e$ come suo punto di equilibrio.  \textbf{Come usare l'ingresso $u(t)$ per rendere il sistema asintoticamente stabile, se $x_e$ non lo è già?} \textbf{È possibile usarlo anche per migliorarne  le prestazioni?} Queste domande sono tipiche del processo di \textit{sintesi di sistemi di controllo.} 
 \begin{lemma}
 	Idealmente, quando un sistema si discosta da un punto di equilibrio, si vuole fare in modo che questo \textit{ritorni ad esso in un tempo} al limite infinito, o che si muova verso \textit{un altro equilibrio}.
 \end{lemma}
\bb
Adesso facciamo un'importante considerazione: \textbf{supponiamo di conoscere, istante per istante, l'intero stato $x(t)$ del sistema.} Avremmo quindi collegato un sensore \textit{a ciascuna componente di $x(t)$:}
\begin{equation}
	y \in \R^n \ \longrightarrow \ y(t) = x(t) 
\end{equation}
Se il sistema non è attualmente stabile, sappiamo che è possibile lavorare al punto di equilibrio $x_e = 0$ e poi estendere quanto calcolato a tutto il sistema. Quando avremo $x_e$ stabile, allora avremo reso stabile \textit{l'intero sistema}. Come fare? Possiamo usare una \textbf{retroazione (statica) dallo stato}, in base alla quale si prende come $u(t)$ non una semplice funzione del tempo, ma una \textbf{funzione dell'intero stato $x(t)$} (o dell'uscita, visto che l'abbiamo direttamente collegata allo stato), a cui sommiamo eventualmente un ulteriore ingresso $v$:
\begin{equation}
	u(t) = K(x(t)) + v(t), \quad K : \R^n \rightarrow \R^m
\end{equation}
Il termine \textbf{statica} della retroazione deriva dal fatto che \textbf{u(t) dipende solamente dallo stato presente, ossia quello rilevato dai sensori al tempo $t$, i.e. $x(t)$.} Versione più evolute della retroazione prevedono dipendenze anche dallo stato passato. Non è questo il caso. La mappa $K$ va dalla dimensione dello stato alla dimensione degli ingressi del sistema ($m$, nel caso generale).
\bb
Attenzione, però: stiamo parlando di sistemi LTI, per cui la mappa $K$ può essere senz'altro presa \textbf{lineare}, per cui scrivibile, per quanto visto nelle pagine precedenti, come \textbf{combinazione lineare} di elementi di $x(t)$. Questo si traduce in una scrittura in forma matriciale:
\begin{equation}
	u(t) = Kx(t) + v(t), \quad K \in \mathcal{M}(m,n,\R).
\end{equation}
Riprendiamo ora la dinamica, e applichiamo l'ingresso appena descritto:
\begin{align}
	\dot x(t) & = Ax(t) + B\paren{Kx(t) + v(t)} = Ax(t) + BKx(t) + Bv(t) \nonumber \\ & = \boxed{\mathbf{(A+BK)}x(t) + Bv(t)}
\end{align}
Abbiamo un sistema detto \textbf{in anello chiuso} (in quanto prende l'intero stato e lo riporta, eventualmente con un'ingresso aggiunto $v(t)$, in ingresso), che non è più forzato; visto che è LTI, \textbf{posso studiare le componenti spettrali della matrice $A+BK$ per caratterizzare il sistema.} La domanda è \textbf{come scegliere la matrice $K$ in modo che il sistema sia asintoticamente stabile?} È evidente che ciò si raggiunge facendo in modo che $A+BK$ abbia \textbf{tutti gli autovalori a parte reale negativa.}
\bb
\begin{defin}{Scelta di $K$ (\textit{pole placement}) e raggiungibilità di un sistema}{}
Vale, ma non è provato qui, che se il sistema descritto dall'equazione di stato $\dot x(t)=Ax(t)+Bu(t)$ è \textbf{raggiungibile}, posso \textbf{allocare arbitrariamente tutti gli autovalori} di $A+BK$, sia in termini di valore che di posizione. Più questi saranno a parte reale negativa, più convergenti saranno i modi che descriveranno la traiettoria del sistema (perfetto!).
\bb
Diciamo che la proprietà di \textit{raggiungibilità}, insieme a quella di \textit{controllabilità} descrivono le possibilità di azione dell'ingresso $u(\cdot)$ al fine di \textit{influenzare la traiettoria di stato.}	
\end{defin}
L'obiettivo è quello di prendere un ingresso \textit{proporzionale all'errore che lo stato ha rispetto all'equilibrio} che si vuole raggiungere. Tutto questo però funziona a patto che sia soddisfatta un'\textbf{ipotesi estremamente forte: siamo capaci di misurare, istante per istante, l'intero stato $x(t)$ del sistema.} Cosa succede se questo non può accadere? Consideriamo due strade percorribili:
\begin{itemize}
	\item \textbf{teoria dei sistemi} \rarr continuare a lavorare nello spazio di stato e ricostruire $x(t)$ da $y(t)$. La copia creata si chiama \textbf{stima dello stato $\bar x(T)$}, e sarà poi utilizzabile nell'equazione dell'ingresso, che diventerà $u(t) = K\bar x(t)$. Per la ricostruzione entrano in gioco sistemi ausiliari di stima chiamati \textbf{osservatori} (il cui funzionamento dipenderà \textit{dall'osservabilità} del sistema in studio). Questo approccio non verrà utilizzato qui;
	\item \textbf{studio di sistemi LTI SISO} \rarr vedremo che in questo caso $(m = p = 1)$, lo stato sarà analizzabile lavorando nel \textbf{dominio di Laplace}.
\end{itemize}

\section{Linearizzazione di sistemi non lineari tempo invarianti (NLTI)}
L'idea alla base della linearizzazione è quella di non dover buttare via tutti i risultati utilissimi visti per l'analisi (e la sintesi) di sistemi LTI (modi, retroazione, stabilità globale mediante studio puntuale...), ma di utilizzarli anche per i NLTI.
\bb
	Supponiamo di avere un sistema non lineare tempo invariante$$\begin{dcases}
	\dot x(t) = f(x(t),u(t)) \\
	y(t) = h(x(t), u(t))
\end{dcases}
 \quad \quad x(0) = x_0$$
	Supponiamo $(x_e,u_e)$ siano una coppia di equilibrio. Allora sappiamo che $f(x_e, u_e) = 0$. Consideriamo ora una traiettoria di stato a partire dallo stato iniziale $x(0) = x_e + \Delta x_0$, e una traiettoria di uscita:
\begin{align*}
\underbrace{x(t) = x_e+\Delta x(t), \quad \quad u(t) = u_e+ \Delta u(t)}_{\textrm{traj. di stato } (x(t), u(t))}, \quad  \quad \underbrace{y(t) = h(x_e, u_e) + \Delta y(t) = y_e + \Delta y(t)}_{\textrm{traj. di uscita}}
\end{align*}
Stiamo cioè considerando traiettorie che non partono più dall'equilibrio, ma da un punto vicino ad esso. Essendo traiettorie, le funzioni di cui sopra risolveranno le equazioni, rispettivamente, di stato e uscita:
\begin{align*}
\begin{dcases}
	\dv{t}(x_e + \Delta x(t)) = f\paren{x_e+\Delta x(t), u_e + \Delta u(t)} \\
	y_e + \Delta y(t) = h(x_e+\Delta x(t), u_e + \Delta u(t))
\end{dcases} = (\star)
\end{align*}
\newpage
Sviluppiamo ora $f$ e $h$ con Taylor per funzioni a due variabili\footnote{Sia $f: A \subseteq \R^2 \rightarrow \R$. Fissato un punto $(x_0, y_0) \in A$, definiamo lo sviluppo in serie Taylor con questo polinomio:
\begin{align*}
	T(x,y) & = f(x_0, y_0) + \pdv{f(x_0, y_0)}{x} (x - x_0)+\pdv{f(x_0, y_0)}{y} (y - y_0) +\\ & + \frac{1}{2} \sparen{\pdv[2]{f(x_0, y_0)}{x} (x - x_0)^2 + \pdv[2]{f(x_0, y_0)}{y} (y - y_0)^2 + \pdv{f(x_0, y_0)}{x}{y} (x - x_0)(y-y_0)} + \textrm{termini ordine sup.}
\end{align*}
}, nel punto $(x_0, u_0) = (x_e, u_e)$, fermandoci alle derivate parziali prime:
\begin{align*}
	f(x_e + \Delta x(t), u_e + \Delta u(t)) & \stackrel{T}{=} \cancel{f(x_e, u_e)} + \pdv{f(x_e, u_e)}{x} (x_e + \Delta x(t) - x_e)+\pdv{f(x_e, u_e)}{u} (u_e + \Delta u(t) - u_e) \\ & =  \pdv{f(x_e, u_e)}{x} \Delta x(t)+\pdv{f(x_e, u_e)}{u} \Delta u(t) + \textrm{termini ord. sup.}
\end{align*}
Analogamente per l'uscita (i passaggi sono soppressi):
\begin{align*}
h(x_e + \Delta x(t), u_e + \Delta u(t)) \stackrel{T}{=} h(x_e, u_e)+\pdv{h(x_e, u_e)}{x} \Delta x(t)+\pdv{h(x_e, u_e)}{u} \Delta u(t) + \textrm{termini ord. sup.}
\end{align*}
Perché ci siamo fermati al primo ordine del polinomio? Ricordiamo l'obiettivo: vogliamo ricondurre le equazioni di un sistema NLTI (che presentano quindi le generali $f,h$) ad equazioni di un sistema LTI. Dobbiamo mettere in conto che per far questo è necessario \textit{accettare un errore di approssimazione} che deriva proprio dal fermarsi prima. Riprendiamo le equazioni del NLTI, e sosituiamo al posto di $f,h$ i polinomi di Taylor che abbiamo appena calcolato:
\begin{align*}
	(\star) = \begin{dcases}
	\dv{t}(x_e + \Delta x(t)) = \pdv{f(x_e, u_e)}{x} \Delta x(t)+\pdv{f(x_e, u_e)}{u} \Delta u(t) \\
	y_e + \Delta y(t) = h(x_e, u_e) + \pdv{h(x_e, u_e)}{x} \Delta x(t)+\pdv{h(x_e, u_e)}{u} \Delta u(t)
\end{dcases} = (\star_2)
\end{align*}
Sviluppiamo la derivata al primo membro dell'equazione di stato, ottenendo $\dot{\Delta x(t)}$, e, sapendo che $y_e = h(x_e, u_e)$ dalla definizione di equilibrio, cancelliamo i termini dalla seconda equazione:
\begin{align*}
	(\star_2) = \begin{dcases}
	\dot{\Delta x(t)} = \pdv{f(x_e, u_e)}{x} \Delta x(t)+\pdv{f(x_e, u_e)}{u} \Delta u(t) \\
	\Delta y(t) = \pdv{h(x_e, u_e)}{x} \Delta x(t)+\pdv{h(x_e, u_e)}{u} \Delta u(t)
\end{dcases} 
\end{align*}
Quella che abbiamo appena scritto è la \textbf{forma di stato di un sistema lineare tempo invariante}! Le espressioni delle derivate parziali sono infatti delle matrici \textit{jacobiane}, ossia contenenti le derivate parziali fatte rispetto a ciascuna variabile di ciascuna componente della funzione specificata:
\begin{align*}
	\pdv{f(x_e, u_e)}{x} = \eval{\begin{bmatrix}
		\pdv{f_1(x, u)}{x_1} & \cdots	& \pdv{f_1(x, u)}{x_n} \\
		\vdots & \ddots & \vdots \\
	\pdv{f_n(x, u)}{x_1} & \cdots & \pdv{f_n(x, u)}{x_n}
	\end{bmatrix}}_{\substack{x=x_e \\ u=u_e}} 
	\quad 
	\pdv{f(x_e, u_e)}{u} = \eval{\begin{bmatrix}
	\pdv{f_1(x, u)}{u_1} & \cdots & 	\pdv{f_1(x, u)}{u_m} \\
		\vdots & \ddots & \vdots \\
	\pdv{f_n(x, u)}{u_1} & \cdots &		\pdv{f_n(x, u)}{u_m}
	\end{bmatrix}}_{\substack{x=x_e \\ u=u_e}} 
\end{align*}
\begin{align*}
	\pdv{h(x_e, u_e)}{x} = \eval{\begin{bmatrix}
		\pdv{h_1(x, u)}{x_1} & \cdots	& \pdv{h_1(x, u)}{x_n} \\
		\vdots & \ddots & \vdots \\
	\pdv{h_p(x, u)}{x_1} & \cdots & 		\pdv{h_p(x, u)}{x_n}
	\end{bmatrix}}_{\substack{x=x_e \\ u=u_e}}  \quad 
	\pdv{f(x_e, u_e)}{u} = \begin{bmatrix}
	\pdv{h_1(x, u)}{u_1} & \cdots & 	\pdv{h_1(x, u)}{u_m} \\
		\vdots & \ddots & \vdots \\
	\pdv{h_p(x, u)}{u_1} & \cdots &		\pdv{h_p(x, u)}{u_m}
	\end{bmatrix}_{\substack{x=x_e \\ u=u_e}} 
\end{align*}
Chiamiamo queste matrici $A_e, B_e, C_e, D_e$, e riscriviamo la dinamica (usando il simbolo di approssimazione perché ricordiamo di aver tolto i termini di ordine superiore dallo sviluppo di Taylor, cosa che inevitabilemnte va a generare un'imprecisione):
\begin{align*}
	\begin{dcases}
		\dot{\Delta x(t)} \approx A_e \Delta x(t) + B_e \Delta u(t) \\
		\Delta y(t) \approx C_e \Delta x(t) + D_e \Delta u(t) 
	\end{dcases}, \quad \Delta x(0) = \Delta x_0
\end{align*}
\newpage
A questo punto, possiamo dare il risultato finale:
\begin{defin}{Linearizzazione di un sistema NLTI nell'intorno ad un equilibrio}{}{}
	Dato un sistema non lineare tempo invariante avente la seguente forma di stato:
\begin{equation*}
	\begin{dcases}
	\dot x(t) = f(x(t),u(t)) \\
	y(t) = h(x(t), u(t))
\end{dcases}
 \quad \quad x(0) = x_0
	\end{equation*}
	Mediante approssimazioni al primo ordine sulle funzioni $f,h$, è possibile \textbf{linearizzarlo nell'intorno di una coppia di equilibrio $(x_e, u_e)$}, in modo che diventi scrivibile in forma matriciale esattamente come un sistema LTI (la $\delta$ è solo \textit{notazione}):
\begin{equation}
\label{eq:lti_da_nlti}
	\begin{dcases}
		\dot{\delta x(t)} = A_e \delta x(t) + B_e \delta u(t) \\
		\delta y(t) = C_e \delta x(t) + D_e \delta u(t)
	\end{dcases}
	\end{equation}
	con $A_e,\cdots, D_e$ matrici \textit{jacobiane} la cui forma è scritta sopra. 
	Vale che:
	\begin{equation}
		(\delta x(t), \delta u(t)), \ t \geq 0
	\end{equation}
	è una \textbf{traiettoria del sistema LTI} descritto alla \eqref{eq:lti_da_nlti}, e si vuole che questa approssimi le variazioni di stato e l'uscita del NLTI con qualcosa che è $x_e$ (o $u_e$) più una componente lineare di \textit{perturbazione} $\delta x(t)$ (o $\delta u(t)$):
\begin{equation}
		x(t) \approx x_e + \delta x(t), \quad \quad u(t) \approx u_e + \delta u(t)
	\end{equation}
	Di conseguenza, è necessario che le variazioni del sistema non lineare rispetto al punto di equilibrio (dove abbiamo linearizzato) siano sufficientemente piccole, in modo da riuscire a \textit{confonderle}. 
	\end{defin}
	
\subsection{Retroazione dallo stato applicata a sistemi NLTI}
Quanto visto nella sezione sulla retroazione dallo stato per sistemi LTI non asintoticamente stabili può essere applicato ad un qualunque sistema NLTI linearizzabile, semplicemente \textbf{agendo sulla sua linearizzazione}. Consideriamo un NLTI e linearizziamolo nell'intorno dell'equilibrio $(x_e, u_e)$:
\begin{equation*}
	\dot x(t) = f(x(t), u(t)) \quad \rightsquigarrow \quad \dot{\delta x(t)} = A_e \delta x(t) +B_e \delta u(t)
\end{equation*}
Possiamo dunque dire che, nelle vicinanze di $(x_e, u_e)$ l'NLTI si approssima al primo ordine con l'LTI di destra. Le matrici $A_e, B_e$ sono le jacobiane calcolate all'equilibrio che abbiamo visto qualche pagina fa. Prendiamo ora il caso che questo LTI non sia stabile; per renderlo stabile (e sappiamo che se lo facciamo in corrispondenza di un punto di equilibrio, varrà per tutti gli altri) possiamo sfruttare il \textit{pole placement} mettendo in gioco la matrice $K$ con l'obiettivo di \textbf{portare $x(t)$ ad $x_e$ in modo approssimato}, i.e. di annullare la perturbazione $\delta x(t)$. Supponendo ancora una volta di \textbf{essere in grado di conoscere l'intero stato istante per istante}, è possibile utilizzare un ingresso del tipo:
\begin{equation}
	\delta u(t) = K\delta x(t) + \delta v(t), \quad K \in \mathcal{M}(m,n,\R).
\end{equation}
Il sistema in anello chiuso sarà dunque descritto da:
\begin{equation}
	\dot{\delta x(t)} = (\mathbf{A_e + B_e K})\delta x(t) + B_e \delta v(t)
\end{equation}
A questo punto, supponendo raggiungibilità, è possibile allocare gli autovalori che mi interessano con l'obiettivo di rendere la matrice in grassetto asintoticamente stabile (che si traduce in componenti spettrali aventi parte reale negativa).
\begin{prop}
	Se il sistema linearizzato nell'intorno di $(x_e, u_e)$ associato al NLTI è asintoticamente stabile, allora il non lineare \textbf{convergerà ad $x_e$}, e quindi sarà \textbf{localmente stabile} in quel punto, \textbf{se parte da un intorno sufficientemente piccolo da $x_e$} stesso.
\end{prop} 
In altre parole, linearizzare un pendolo nell'intorno di un equilibrio $(\pi, 0)$ e renderlo asintoticamente stabile lì, non renderà il NLTI associato stabile, se quest'ultimo parte da $(0,0)$, perché siamo ben fuori dall'intorno in cui la linearizzazione ha validità.
\newpage
Vediamo infine come riportare al sistema NLTI l'\textbf{equazione dell'ingresso con feedback} che prima abbiamo definito per il linearizzato. In particolare, sapendo che $u(t) \approx u_e + \delta u(t)$ e che $x(t) \approx x_e + \delta x(t)$, scriviamo:
\begin{align}
	u(t) = u_e + (K\delta x(t) + \delta v(t)) = u_e + K(x(t) - x_e) + \delta v(t),
\end{align}
con $K$ progettata sul sistema linearizzato.
\resource{0.5}{linearizz}{Schema di un NLTI reso localmente stabile mediante feedback dallo stato e linearizzazione.}
\bb
\begin{prop}
	Se la linearizzazione di un NLTI ha una matrice $A$ avente almeno una componente spettrale a parte reale positiva (o nulla e non semplice), allora non è stabile in quel punto e non lo sarà nemmeno (localmente) il non lineare associato.
\end{prop}

\section{Esempio e note aggiuntive}

\begin{esem}
	Vediamo ora un esempio di linearizzazione e stabilizzazione di un sistema NLTI. Ricordiamo l'equazione di stato del pendolo:
	\begin{equation*}
		\dot x(t) = \begin{bmatrix}
			\dot x_1(t) \\ \dot x_2 (t)
	\end{bmatrix} = \begin{bmatrix}
		x_2(t)\\
		-\frac{g}{l} \sin(x_1(t)) - \frac{b}{Ml^2} x_2(t) + \frac{1}{Ml^2}u(t)
	\end{bmatrix}
	\end{equation*}
	Consideriamo un equilibrio per questo sistema, della forma $\begin{pmatrix}\begin{pmatrix}
		x_{1e} \\ x_{2e}
	\end{pmatrix}, u_e
	\end{pmatrix}$. Vale che un punto di eq. annulla la funzione $f$, per cui scriviamo:
	\begin{equation*}
	\begin{bmatrix}
		x_{2e}\\
		-\frac{g}{l} \sin(x_{1e}) - \frac{b}{Ml^2} x_{2e} + \frac{1}{Ml^2}u_e
	\end{bmatrix} = \begin{bmatrix}
		0 \\ 0 
	\end{bmatrix} \Rightarrow \begin{dcases}
	u_e = - Mlg \sin(x_{1e})\\
		x_{2e} = 0
	\end{dcases}
	\end{equation*}
	Uno stato di equilibrio per il pendolo comporta quindi sempre $x_{2e} = 0$, che rappresenta la velocità angolare. Questo ha senso! La coppia in questione è quindi:
	\begin{equation*}
	\begin{pmatrix}
	\begin{pmatrix}
			x_{1e} \\ 0
	\end{pmatrix}, -Mlg\sin(x_{1e})
	\end{pmatrix}
	\end{equation*}
	Esempi di equilibrio sono la posizione verticale in basso $(x_e = \mathbf{0}_{\R^2})$ oppure verticale $(x_{1e} = \pi)$. In entrambi i casi si ha $u_e = 0$, il che significa che non è necessario applicare un ingresso per mantenerlo in equilibrio (ed è effettivamente sensato). In una qualsiasi altra posizione, l'effetto di $g$ doveva essere controbilanciato da una $u(t)$, per mantenere lo stato costante (alrimenti cadrebbe).
Linearizziamo ora il NLTI intorno a questo equilibrio. Otteniamo la scrittura
\begin{align*}
	\dot{\delta x(t)} = A_e \delta x(t) + B_e \delta u(t) & = \eval{\begin{bmatrix}
		\pdv{f_1(x, u)}{x_1} & \pdv{f_1(x, u)}{x_2} \\
		\pdv{f_2(x, u)}{x_1} & \pdv{f_2(x, u)}{x_2}
	\end{bmatrix}}_{\substack{x=x_e \\ u=u_e}}  \delta x(t) + \eval{\begin{bmatrix}
		\pdv{f_1(x, u)}{u} \\ \pdv{f_2(x, u)}{u}
	\end{bmatrix}}_{\substack{x=x_e \\ u=u_e}}  \delta u(t) \\ & = \eval{\begin{bmatrix}
		0 & 1 \\ -\frac{g}{l} \cos(x_{1}(t)) & -\frac{b}{Ml^2} 
	\end{bmatrix}}_{\substack{x=x_e \\ u=u_e}}  \delta x(t) + \eval{\begin{bmatrix}
		0 \\ \frac{1}{Ml^2}
	\end{bmatrix}}_{\substack{x=x_e \\ u=u_e}} \delta u(t) \\ & = \begin{bmatrix}
		0 & 1 \\ -\frac{g}{l} \cos(x_{1e}) & -\frac{b}{Ml^2} 
	\end{bmatrix}  \delta x(t) + \begin{bmatrix}
		0 \\ \frac{1}{Ml^2}
	\end{bmatrix} \delta u(t) 
\end{align*}
Analizziamo la stabilità di questo sistema lineare (che approssima al primo ordine il non lineare di partenza, nell'intorno di un punto di equilibrio), e nel caso procediamo con il pole placement. Consideriamo due casi di equilibrio:
\subsubsection{Caso $((x_{1e}, x_{2e}), u_e) = ((0,0), 0)$}
Il linearizzato intorno a questo punto ha la forma:
\begin{align*}
	\dot{\delta x(t)} = \begin{bmatrix}
 0 & 1 \\ -\frac{g}{l} & - \frac{b}{Ml^2}	
 \end{bmatrix} \delta x(t) + \begin{bmatrix}
 	0 \\ \frac{1}{Ml^2}
 \end{bmatrix} \delta u(t) \quad \rightarrow \quad 
	\det(A_e - \lambda I) = 0 \rightarrow \cdots \rightarrow \ \Re{\lambda_i} < 0 \ \forall i
\end{align*}
Avendo componenti spettrali tutte a parte reale negativa, il sistema LTI è asintoticamente stabile. Questo si traduce in stabilità \textbf{locale} per il NLTI, per variazioni di posizione dal punto di equilibrio sufficientemente piccole.

\subsubsection{Caso $((x_{1e}, x_{2e}), u_e) = ((\pi,0), 0)$}
\begin{align*}
	\dot{\delta x(t)} = \begin{bmatrix}
 0 & 1 \\ \frac{g}{l} & - \frac{b}{Ml^2}	
 \end{bmatrix} \delta x(t) + \begin{bmatrix}
 	0 \\ \frac{1}{Ml^2}
 \end{bmatrix} \delta u(t) \quad \rightarrow \quad 
	\det(A_e - \lambda I) = 0 \rightarrow \cdots \rightarrow \ \exists \lambda_i \ : \ \Re{\lambda_i} > 0
\end{align*}
per cui LTI \textbf{non è} asintoticamente stabile, e occorre passare al pole placement (matrice $K$). Non faremo mai questo calcolo a mano, ma utilizzeremo sempre MATLAB:

\begin{lstlisting}[
  style = Matlab-editor,
  basicstyle = \mlttfamily]
  % [ ... ]
  K = place(A_e, B_e, [-0.5; -0.6]);
  K = -K; % Matlab calcola K considerando A-BK, noi vogliamo A+BK
  
  input = @(t,x) K*(x-x_e);
  
  dyn = @(t, x) [x(2); -g/l*sin(x(1)) - b/(m*l^2)*x(2) + inp(t,x)/(m*l^2)];
  [time, traj] = ode45(dyn, interv, x0);
\end{lstlisting}
Dunque ML calcola, mediante la funzione \texttt{place} una matrice di feedback dallo stato $K$ tale che gli \textbf{autovalori di $A-BK$ siano quelli specificati nel vettore passato come terzo parametro della funzione}. A questo punto, bisogna effettivamente applicare il feedback considerando un input $$u(t) = K(x(t)) \ \textrm{che nel nostro caso diventa} \ \boxed{u(t) =K(\delta x(t)) = K(x(t) - x_e)}$$
Possiamo verificare che effettivamente gli autovalori di $A+BK$ siano quelli scelti da noi mediante la funzione \texttt{eig(A+BK)}. È importante, quindi, l'istruzione \texttt{K = -K}! La simulazione darà un esito di stabilità a patto che il NLTI parta da un punto sufficientemente vicino all'equilibrio che abbiamo utilizzato per la linearizzazione (e il pole placement). Se questo non accade, il sistema NLTI \textbf{non} sarà stabile!

\subsubsection{Caso extra in cui $b = 0$ e punto di equilibrio generico $((x_{1e},x_{2e}),u_e)$}
Il nostro linearizzato sarà:
\begin{equation*}
	\dot{\delta x(t)} = \begin{bmatrix}
 0 & 1 \\ -\frac{g}{l}\cos(x_{1e}) & 0	
 \end{bmatrix} \delta x(t) + \begin{bmatrix}
 	0 \\ \frac{1}{Ml^2}
 \end{bmatrix} \delta u(t) \quad \rightarrow \quad 
	\det(A_e - \lambda I) = 0 \rightarrow \ \lambda = \pm j \sqrt{\frac{g}{l}  \cos(x_{1e})} 
\end{equation*}
Notiamo che nei due equilibri considerati sopra si ha una situazione di \textbf{stabiità marginale} dell'LTI, in quanto gli autovalori sono \textbf{semplici e a parte reale nulla} (ricordiamo: questo è ancora un caso valido, visto che ho supposto la semplicità!). Non appena insorge una perturbazione (cosa perfettamente plausibile, visto che la linearizzazione approssim soltanto il NLTI) si ha divergenza, e il sistema di fatto oscillerà all'infinito senza mai fermarsi. 
\end{esem}

\subsection{Controllo ottimo}
Non abbiamo per niente considerato, in queste pagine, quali performance otteniamo quando applichiamo un particolare tipo di controllo automatico ad un sistema dinamico, e nemmeno il costo dell'implementazione. Progettare per il raggiungimento di certi obiettivi operando certe misure di controllo di costo (che non è necessariamente economico) va sotto il nome di \textbf{controllo ottimo}, che in questo corso non verrà affrontato in generale.

% !TEX root = ./main.tex

\part{Analisi nel dominio di Laplace}

\chapter{Laplace $\mathcal{L}\sparen{\cdot}$}
Come già visto in altri ambiti con Steinmetz e Fourier, anche in questo corso può rivelarsi utile spostare un problema dal dominio temporale ad un altro, risolverlo lì, e riportare la soluzione di nuovo nel tempo.
\resource{0.5}{laplace}{Risoluzione nel dominio di Laplace}
I concetti che vedremo faranno forte uso dei numeri complessi, che però \textbf{non} verranno ripresi qui perché sono sempre le solite cose (rappresentazione cartesiana, polare, esponenziale, come passare dall'una all'altra, proprietà varie). Rimando alla dispensa di TLC-1 dove queste cose sono trattate di più.
\section{Trasformata}
\begin{defin}{}{}
Sia $f: \R \rightarrow \C$ (in questo corso tipicamente $f: \R \rightarrow \R$); sia $s = \sigma + j\omega \in \C$. Definiamo la \textbf{trasformata di Laplace} (se esiste) come:
\begin{equation}
	\lt{f(t)} = F(s) = \ltint{f(t)}{-st}{t}
\end{equation} 
Ci muoviamo, quindi, dal dominio del tempo, al \textbf{dominio di Laplace}: $f(t) \xrightarrow{\Lap} F(s)$. Notiamo gli estremi di integrazione: Laplace non è definita, come Fourier, su tutto l'asse reale, ma solo sul \textbf{semiasse positivo} $[0^-, \pinf]$. Lo $0^-$ è incluso perché bisogna tenere conto, in fase di trasformazione, di eventuali \textbf{impulsi in $t = 0$ della $f$}. La funzione integranda è complesso, per cui l'integrale stesso (e dunque $F(s)$) è complesso. \bb Dato il dominio di definizione, si può dire che l'operazione di trasformazione \textbf{non è biunivoca}, i.e. non è vero che ad ogni $f$ corrisponde una e una sola $F$. Potrei prendere delle funzioni del tutto diverse tra loro nell'intervallo $\R^-$, ma poi identiche in $\R^+$: queste avranno la stessa trasformata. Per ottenere biunivocità, considereremo  sempre \textbf{funzioni definite nel semiasse positivo di $\R$, e nulle altrove} (dette \textbf{funzioni causali)}:
\begin{equation}
	f(t) = 0 \quad \forall t < 0^-
\end{equation}
Supponendo di lavorare con sistemi \textit{tempo invarianti}, questa restrizione non comporta perdità di generalità: posso traslare una qualsiasi funzione TI di un certo valore temporale $t_0$ per riportarla completamente nel dominio $[0^-, \pinf]$, e si comporterà identicamente a come si comporterebbe se non fosse traslata. 
\end{defin}

\begin{prop}
Si può far vedere che:
\begin{equation}
	\exists \bar \sigma \ : \ \forall s = \sigma + j \omega \ \textrm{avente} \ \sigma > \bar \sigma \ \longrightarrow \ F(s) = \int \cdots \dd{t} \ \textrm{converge}.
\end{equation}
Esiste cioè un'ascissa $\bar \sigma$ sul piano complesso, detta \textbf{di convergenza}, a destra della quale si trovano tutti i complessi che fanno \textbf{convergere l'integrale} della trasformazione di Laplace. Dunque, \textbf{la trasformata esiste a destra di quest'ascissa} (i.e. nel semipiano $\Re{s} > \bar \sigma$), (ma sarà possibile estendere l'esistenza anche a sinistra di $\bar \sigma$, i.e. per $\Re{s} \leq \bar \sigma$).
\end{prop}

\subsection{Trasformata come rapporto di polinomi}
Vedremo nel corso $f(t)$ aventi sempre come trasformata un \textbf{rapporto di polinomi} (a coefficienti reali, se $f$ è reale):
\begin{equation}
\label{lt_rap}
	F(s) = \frac{N(s)}{D(s)}
\end{equation} 
Le radici del numeratore ($N(s) = 0$) sono dette \textbf{zeri}, quelle del denominatore, \textbf{poli}. Valgono i seguenti:
\begin{them}
(del valore iniziale). Se $f(t)$ è una funzione reale avente trasformata $F(s)$ esprimibile come in \eqref{lt_rap}, e \textbf{il grado di $D(s)$ è maggiore di quello di $N(s)$}, allora
\begin{equation}
\llimit{s}{\pinf}{sF(s)} = f(0).
\end{equation}
\end{them}

\begin{them} (del valore finale). Se alle ipotesi del teorema di cui sopra aggiungiamo anche che \textbf{il denominatore ha tutti i poli nulli o a parte reale negativa}, allora vale
\begin{equation}
\llimit{s}{0}{sF(s)} = \llimit{t}{\pinf}{f(t)}
\end{equation} 
\end{them}
Grazie a questi possiamo determinare i \textbf{valori asintotici }iniziali e finali di una funzione nel tempo \textbf{partendo dalla sua trasformata} di Laplace. Le dimostrazioni sono omesse.

\section{Antitrasformata}
\begin{defin}{}{}
Per tornare nel dominio del tempo ($F(s) \xrightarrow{\Lap^{-1}} f(t)$) usiamo la formula di \textbf{antitrasformazione}:
\begin{equation}
	f(t) = \lat{F(s)} = \latint{F(s)}{st}{s}, \quad \quad \sigma > \bar \sigma.
\end{equation}
Notiamo la restrizione per $\sigma$: abbiamo detto che $F(s)$ esiste a destra di $\bar \sigma$, quindi l'antitrasformazione (che integra $F(s)$ stesso) dovrà rispettare questo requisito. Questa formula non verrà mai utilizzata praticamente.
\end{defin}


\section{Proprietà}
\begin{itemize}
	\item \textbf{Linearità} \rarr \boxed{\lt{\alpha f(t) + \beta g(t)} = \alpha \lt{f(t)} + \beta \lt{g(t)}, \ \forall \alpha, \beta \in \R} \\ \\ si dimostra sfruttando la linearità dell'operatore integrale;
	\item \textbf{Traslazione temporale} \rarr \boxed{\lt{f(t-\tau)} = \lt{f(t)}e^{-s\tau}, \ \forall \tau > 0} : \\ \\ la trasformata di una funzione traslata sarà la trasformata della funzione non traslata, più un termine moltiplicativo che dipende dal quantitativo di traslazione.
	\begin{proof}
	Usiamo la definizione e consideriamo un cambio di variabile $y = t-\tau$, da cui $\dd{y} = \dd{t}$:
	\begin{align*}
		\lt{f(t-\tau)} = F(s) & = \ltint{f(t-\tau)}{-st}{t} = \int_{-\tau}^{\pinf} f(y)e^{-s(y+\tau)} \dd{y} = (\star)
	\end{align*}
	ma abbiamo detto di voler considerare funzioni nulle nel semiasse negativo, per cui possiamo cambiare l'estremo di integrazione:
	\begin{equation*}
		(\star) = \int_{0^-}^{\pinf} f(y)e^{-sy}\dd{y} e^{-s\tau} = \lt{f(t)}e^{-s\tau}.
	\end{equation*}
	\end{proof}
	\item \textbf{Traslazione nel dominio complesso} \rarr \boxed{\lt{e^{\alpha t}f(t)} = F(s-\alpha), \ \forall \alpha \in \C} : \\ \\ la traslazione nel dominio di Laplace  di un certo complesso $\alpha$ corrisponde ad un termine moltiplicativo nel dominio del tempo che dipende dalla translazione;
	\begin{proof}
		\begin{equation*}
		\lt{e^{\alpha t}f(t)}  = \ltint{e^{\alpha t}f(t)}{-st}{t} = \ltint{f(t)}{-(s-\alpha)t}{t} = F(s-\alpha) 		
		\end{equation*}
	\end{proof}
	\item \textbf{Derivazione (nel tempo)} \rarr \boxed{\lt{\dv{f(t)}{t}} = sF(s)-f(0^-)}
		\begin{proof}
	Usiamo la definizione di trasformata e applichiamo l'integrazione per parti:
	\begin{align*}
		\lt{\dv{f(t)}{t}} & = \ltint{\dv{f(t)}{t}}{-st}{t}	 = \eval{f(t)e^{-st}}_{0^-}^{\pinf} - \int_{0^-}^{\pinf} f(t)(-s)e^{-st} \dd{t} \\ & = 0 - f(0^-) + \int_{0^-}^{\pinf} sf(t)e^{-st} \dd{t} = - f(0^-) + s \int_{0^-}^{\pinf} f(t)e^{-st} \dd{t} \\ & = - f(0^-) + s \lt{f(t)} = sF(s) - f(0^-).
	\end{align*}
	\end{proof}
	\item \textbf{Derivazione (nel tempo) generica} \rarr \boxed{\lt{\dv[n]{f(t)}{t}} = s^{n} F(s) - \sum_{i=1}^{n} s^{n-i} \eval{\dv[i-1]{f(t)}{t}}_{t=0^-}} \footnote{
	Verifichiamo che è valida per la derivata prima:
	\begin{equation*}
		\lt{\dv{f(t)}{t}} = s^1 F(s) - \sum_{i=1}^1 \cdots = s F(s) - s^{0} \eval{f(t)}_{0^-} = sF(s)-f(0^-).
	\end{equation*}}
	\item \textbf{Integrazione (nel tempo)} \rarr \boxed{\lt{\int_0^t f(\tau) \dd{\tau}} = \frac{F(s)}{s}}
\item \textbf{Convoluzione (nel tempo)} \rarr \boxed{\lt{f_1(t) \circledast f_2(t)} = \lt{\int_0^t f_1(t-\tau)f_2(\tau) \dd{\tau}} =  F_1(s)F_2(s)} : \\ \\ il \textbf{prodotto di convoluzione} è definito come l'integrale del prodotto tra una funzione ferma e una seconda che trasla sulla prima. NB: supponiamo che le funzioni $f_1,f_2$ siano nulle per $t < 0$. Nel dominio di Laplace, questo si trasforma in un prodotto semplice tra le trasformate.
\end{itemize}

\section{Trasformate di segnali elementari}

\resource{0.6}{lap_elem}{Segnali elementari (1)}
\resource{0.6}{lap_elem2}{Segnali elementari (2)}
\bb
\textbf{Osservazione:} è importante focalizzarci sul gradino di Heaviside $1(t)$, perché è quella funzione che, moltiplicata per una qualsiasi altra $f(t)$, la rende \textbf{causale}, i.e. nulla per $t < 0$ e inalterata per $t \geq 0$. Poiché, come già detto, vogliamo avere a che fare solo con robe di questo tipo, qualunque funzione potremmo vederla come il prodotto della stessa con il gradino unitario. Un esempio immediato è la \textbf{funzione rampa} $t1(t)$ di cui si riporta sopra la trasformata: nulla fino a $0$ da sinistra, poi lineare $y=t$. 

\chapter{Sistemi LTI nel dominio di Laplace}
Prima di analizzare i sistemi LTI in esclusiva, riportiamo questo risultato che si applica in generale a tutti i sistemi \textbf{lineari}.
\begin{prop}
Abbiamo visto che, per un sistema lineare descritto nel dominio dei tempi, è possibile scrivere traiettoria di stato ed uscita come una \textbf{somma di un'evoluzione libera ed una forzata} (vedi \eqref{sec:traj_somma_evoluz}). Anche nel dominio di Laplace questo è possibile. In particolare, vale:
\begin{equation}
X(s) = X_L(s) + X_F(s), \quad \quad Y(s) = Y_L(s) + Y_F(s)
\end{equation}
\end{prop}

\section{Equazioni delle trasformate delle traiettorie}
Se consideriamo un sistema LTI, è possibile \textbf{calcolare le equazioni esplicite della trasformata della traiettoria di stato e di uscita}, evidenziando le due evoluzioni che la compongono, analogamente a quanto fatto in \eqref{subsec:eq_traj_lti_scalare_gen}. Analizziamo direttamente il caso generale (i.e.  $x \in \R^n, y \in \R^p, u \in \R^m$): 

%\begin{defin}{}{}
\begin{equation*}
\begin{dcases}
\dot x(t) = Ax(t) + Bu(t) \\
y(t) = Cx(t) + Du(t)
\end{dcases}, \quad x(0) = x_0
\end{equation*}
Definiamo ora le \textbf{trasformate di Laplace} di stato, ingresso e uscita:
\begin{equation}
X(s) \coloneqq \lt{x(t)}, \quad \quad Y(s) \coloneqq \lt{y(t)}, \quad \quad U(s) \coloneqq \lt{u(t)}
\end{equation}
Trasformiamo entrambi i membri delle equazioni, sfruttando le proprietà di linearità e derivazione:
\begin{equation*}
\begin{dcases}
\lt{\dot x(t)} = sX(s) - x(0) = AX(s) + BU(s)\\
Y(s) = CX(s) + DU(s) 
\end{dcases} \rightarrow 
\begin{dcases}
sX(s) - AX(s) = x(0) + BU(s) \\
Y(s) = CX(s) + DU(s) 
\end{dcases} = (\star)
\end{equation*}
Raccogliamo al primo membro e moltiplichiamo $s$ per la matrice identica ($n \times n$) $I$, in modo da rendere matematicamente possibile una differenza di matrici\footnote{
Moltiplicare uno scalare $k$ per una matrice $M$ equivale a moltiplicare $k$ per l'identica $I$, e poi moltiplicare per $M$.}. Esplicitiamo poi $X(s)$ dalla prima equazione:

\begin{equation*}
(\star) \rightarrow
\begin{dcases}
(sI-A)X(s) = x_0 + BU(s) \\
Y(s) = CX(s) + DU(s) 
\end{dcases} \rightarrow \begin{dcases}
X(s) = (sI-A)^{-1}x_0 + (sI-A)^{-1}BU(s) \\
Y(s) = CX(s) + DU(s) 
\end{dcases}
\end{equation*}
Sostituendo l'espressione della $X(s)$ all'interno dell'equazione di uscita, si ha
\begin{center}
\boxed{
\begin{tabular}{ccccccc}
    $X(s)$ & $=$ & \parbox{2cm}{\centering
    $(sI-A)^{-1}$} &
  $x_0$ & $+$ & \parbox{3cm}{\centering
    $(sI-A)^{-1}B$} & $U(s)$ \\[0.13cm]
    $Y(s)$ & $=$ & \parbox{2cm}{\centering
    $C(sI-A)^{-1}$} &
  $x_0$ & $+$ & \parbox{3cm}{\centering
    $\sparen{C(sI-A)^{-1}B + D}$} & $U(s)$ \\
\end{tabular}
}
\end{center}
\subsubsection{Trasformate dell'evoluzione libera e forzata per stato ed uscita}
\begin{align}
X_L(s) = (sI-A)^{-1}x_0, \quad & \quad Y_L(s) = C(sI-A)^{-1}x_0 \\
\label{eq:forz}
X_F(s) = (sI-A)^{-1} BU(s), \quad & \quad Y_F(s) = \sparen{C(sI-A)^{-1}B + D}U(s)
\end{align}
%\end{defin}

Vediamo prima qualche risultato che ci tornerà utile in seguito.
\section{Nozioni di calcolo matriciale}
\begin{prop}
Presa una generica matrice a valori complessi $M \in \M{m}{n}{\C}$, definiamo la \textbf{matrice aggiunta} come:
\begin{equation}
M^\dag = \paren{\hat M}^T\end{equation}
i.e. la trasposta dei \textbf{complementi algebrici} di $M$. Il complemento algebrico (o \textit{cofattore}) di un generico elemento $a_{ij} \in M$ è definito come:
\begin{equation}
\hat{M}_{ij} = \cof{a_{ij}} = (-1)^{i+j} \det(M_{ij})
\end{equation}
ossia il determinante di $M$ alla quale è soppressa la $i$-esima riga e la $j$-esima colonna, preso positivo se $i+j$ è pari, dispari altrimenti. 

\end{prop}
\begin{prop}
Il \textbf{metodo di Laplace} per il calcolo del determinante di una \textbf{generica matrice $M$ quadrata} si basa sui complementi algebrici. In particolare, \textbf{fissata una riga $i$ (o una colonna $j$) qualsiasi di $M$}, il determinante è pari a:
\begin{equation} \underbrace{\sum_{j=1}^n a_{ij} \cof{a_{ij}}}_{\textrm{riga $i$ fissata}} = \det(A) = \underbrace{\sum_{i=1}^n a_{ij} \cof{a_{ij}}}_{\textrm{colonna $j$ fissata}}
\end{equation}
Si sommano cioè i prodotti fra gli \textbf{elementi della riga (o della colonna) scelta} - quindi si itera sulla colonna se fisso una riga, su una riga se fisso una colonna - e \textbf{i rispettivi complementi algebrici}. Il metodo ha un approccio ricorsivo fin quando non si arriva ad una $3\times 3$, per la quale si può usare Sarrus, o $2\times 2$, per la quale invece esiste una formula specifica.
\end{prop}
\begin{prop}
Usando la definizione di matrice aggiunta, possiamo calcolare l'\textbf{inversa} di una generica matrice \textbf{quadrata} $M \in \M{n}{n}{\C}$ usando la seguente relazione:
\begin{equation}
A^{-1} = \frac{A^\dag}{\det(A)}
\end{equation}
\end{prop}

\begin{esem}
Consideriamo $A$ e calcoliamo $(sI-A)^{-1}$. Abbiamo bisogno del suo determinante e della sua matrice aggiunta.
\begin{equation*}
A = \begin{bmatrix}
0 & 1 \\ \alpha & \beta 
\end{bmatrix} \quad\longrightarrow \quad sI-A = \begin{bmatrix}
s & -1 \\ -\alpha & s- \beta
\end{bmatrix}
\end{equation*}

\subsubsection{Determinante di $sI-A$}
\begin{equation*}
\det(sI-A) = s(s-\beta) -\alpha = s^2 - \beta s -\alpha
\end{equation*}
\subsubsection{Matrice aggiunta di $sI-A$}
Calcoliamo la matrice di cofattori (solo il primo viene scritto esplicitamente) e trasponiamola (chiamo $sI-A = K$ per brevità in questi passaggi):
\begin{align*}
\cof{K_{11}} = (-1)^{1+1} \det\paren{\begin{bmatrix}
s-\beta
\end{bmatrix}} = s-\beta \quad & \quad
\cof{K_{12}} = -(-\alpha) = \alpha \\
\cof{K_{21}} = -(-1) = 1 \quad & \quad \cof{K_{22}} = s
\end{align*}
da cui:
\begin{equation*}
(sI-A)^\dag = \paren{\hat{sI-A}}^T = \begin{bmatrix}
s-\beta & \alpha \\ 1 & s
\end{bmatrix}^T = \begin{bmatrix}
s-\beta & 1 \\
\alpha & s
\end{bmatrix}
\end{equation*}
\subsubsection{Calcolo della matrice inversa}
\begin{equation*}
(sI-A)^{-1} = \frac{(sI-A)^\dag}{\det(sI-A)} = \begin{bmatrix}
\displaystyle\frac{s-\beta}{s^2-\beta s -\alpha} & \displaystyle\frac{1}{s^2-\beta s -\alpha} \\[0.4cm] 
\displaystyle\frac{\alpha}{s^2-\beta s -\alpha} & \displaystyle\frac{s}{s^2-\beta s -\alpha}
\end{bmatrix}
\end{equation*}
\end{esem}
\newpage
\section{Funzione di trasferimento}
\begin{defin}{}{}
Recuperiamo la definizione dell'\textbf{evoluzione forzata dell'uscita} scritta in \eqref{eq:forz}. Chiamiamo  \textbf{funzione di trasferimento} la matrice:
\begin{equation}
\label{eq:_trasf}
G(s) = C(sI-A)^{-1}B + D
\end{equation}
Nel caso in cui il sistema sia SISO, avremmo $B$ $(n\times 1)$, $C$ $(1 \times n)$, $D$ $(1 \times 1)$, per cui $G(s)$ è scalare. La nuova rappresentazione ingresso-uscita (forzata) diventa quindi, nel caso generale:
\begin{equation}
\label{eq:gs_evoforz_y}
Y_F(s) = G(s)U(s)
\end{equation}
Se invece supponiamo di avere un sistema senza evoluzione libera, quindi con stato iniziale \textit{nullo}, l'equazione di sopra diventa l'espressione totale della (trasformata della) traiettoria di uscita. Da questa si ricava la relazione tra la funzione di trasferimento e le trasformate di uscita/ingresso:
\begin{equation}
	x(0) = x_0 = 0 \quad \rightarrow \quad Y(s) = G(s)U(s) \quad \textrm{da cui} \quad \boxed{G(s) = \frac{Y(s)}{U(s)}}
\end{equation}
Conoscere $G(s)$ è quindi potentissimo, in quanto mi definisce da sola il comportamento del sistema in analisi, permettendomi immediatamente di calcolare l'uscita corrispondente ad un ingresso.
\bb
Facendo uso dei richiami matriciali di inizio sezione, definiamo operativamente la funzione di trasferimento, a partire dalla \eqref{eq:_trasf}:
\begin{equation}
G(s) = C \frac{(sI-A)^\dag}{\det(sI-A)}B+D
\end{equation} 
\end{defin}
Notiamo che il determinante è un polinomio, dunque quel rapporto non è altro che una moltiplicazione di una matrice $n\times n$ per uno scalare. Il risultato dell'inversa è dunque ancora una $n\times n$.
\begin{itemize}
\item caso generale \rarr $C$ è una $p \times n$, e $B$ è una $n \times m$, dunque il prodotto ordinato tra le tre dà in uscita una $p \times m$, che corrisponde alle dimensioni di $D$. $G$ è dunque una \textbf{matrice} i cui elementi sono rapporti di polinomi, giusto? Sì, è giusto, ho chiesto al prof;
\item caso SISO \rarr $C$ è una $1\times n$, e $B$ è una $n\times 1$, dunque il prodotto ordinato dà uno scalare, che ancora corrisponde alle dimensioni di $D$. Vediamo meglio:
\end{itemize}

\subsection{$G(s)$ razionale fratta e fattorizzazioni (sistemi LTI SISO)}
Dunque, nel caso SISO, $G$ è scalare, ed in particolare, è una \textbf{funzione razionale fratta} di questo tipo:
\begin{equation}
\label{eq:trasf_polinomi}
\textrm{se SISO} \quad \rightarrow \quad G(s) = \frac{N(s)}{D(s)} = \frac{\beta_\nu s^\nu + \beta_{\nu-1} s^{\nu -1} + \cdots + \beta_1 s + \beta_0}{s^\nu + \alpha_{\nu - 1}s^{\nu -1} + \cdots + \alpha_1 s + \alpha_0} 
\end{equation}
Vediamo alcune proprietà:
\begin{itemize}
\item vale, in genere, $\deg(N) \leq \deg(D)$; sia per denominatore, che per numeratore, si utilizza lo stesso pedice/esponente $\nu$, ad indicare che il grado di quest'ultimo potrà al massimo essere $\nu$ (i.e. quello del denominatore). Potrà ovviamente essere inferiore, se i coefficienti $\beta$ sono nulli. Il denominatore però sarà sempre di grado $\nu$ (notare infatti l'assenza di coefficiente);  
\item chiamiamo \textbf{grado relativo} $r$ la differenza tra il grado di $D$ e quello di $N$.
\end{itemize}
Questo concetto è strettamente legato al concetto di sistema \textit{proprio/strettamente proprio}. Ricordiamo che in un sistema \textit{strettamente proprio} l'uscita \textit{non dipende} direttamente dall'ingresso, dunque, nel caso generale, è della forma: $y(t) = h(x(t),t)$.  Enunciamo la relazione:
\begin{defin}{}{}
A partire dalla funzione di trasferimento di un sistema LTI SISO (quindi $G$ è scalare):
\begin{equation*}
G(s) = C \frac{(sI-A)^\dagger}{\det(sI-A)}B+D
\end{equation*}
vale che \textbf{se il sistema è proprio, ossia l'equazione dell'uscita è del tipo $y = Cx+Du$, con matrice $D \neq 0$, allora grado di numeratore e denominatore della $G$ sono uguali:}
\begin{equation}
\textrm{se} \ D \neq 0 \ \textrm{(sistema proprio)} \quad \rightarrow \quad \deg(N(s)) = \deg(D(s)) \ \textrm{della} \ G(s)
\end{equation}
Sotto queste condizioni, quindi, il grado relativo è 0. La dimostrazione, non rigorosa, è su carta.
\end{defin}
\begin{itemize}
\item Essendo $N,D$ polinomi \textbf{a coefficienti reali}, poli e zeri saranno \textbf{o reali o complessi coniugati}. Inoltre, \textbf{i poli sono gli autovalori di $A$}, essendo $D$   composto dalla quantità scalare $\det(sI-A)$, che rappresenta esattamente il \textit{polinomio caratteristico} di $A$.
\end{itemize}
\resource{0.4}{poli_zeri_gauss}{Rappresentazione di poli e zeri sul piano di Gauss.}
\bb
Approfondiamo ora meglio questo discorso di poli e zeri. È ammesso scrivere un rapporto di polinomi in funzioni delle sue radici. Applicando questo discorso 
alla \eqref{eq:trasf_polinomi}, eseguiamo questa riscrittura in funzione dei suoi \textit{poli} e  \textit{zeri}:

\begin{defin}{Prima e seconda forma fattorizzata}{}
Per i sistemi LTI SISO, valgono le seguenti:
\begin{equation}
\label{eq:prima_fatt}
G(s) = \frac{\rho \prod_i (s+z_i)\prod_i (s^2+2\zeta \alpha_{n,i}s+\alpha^2_{n,i})}{s^g \prod_i(s+p_i) \prod_i(s^2+2\xi_i \omega_{n,i}s + \omega^2_{n,i})}
\end{equation}
con $\rho$ costante di trasferimento, $g$ tipo, $-z_i$ zeri reali, $-p_i$ poli reali e poi grandezze legate agli zeri e ai poli \textbf{complessi coniugati}, tra cui pulsazioni naturali $\alpha_{n,i}$ e $\omega_{n,i}$ (entrambi definiti $>0$) e smorzamenti $\zeta_{n,i}$ e $\xi_{n,i}$ (entrambi definiti in $]-1,1[$ ).
\bb
La seconda forma è del tutto analoga alla prima, ma mette in evidenza grandezze differenti e tornerà utile successivamente, quando si andrà a studiare il comportamento \textbf{in frequenza}:
\begin{equation}
\label{eq:sec_fatt}
G(s) = \frac{\mu\prod_i (1+\tau_i s) \prod_i \paren{1+\frac{2\zeta_i}{\alpha_{n,i}}s + \frac{s^2}{\alpha^2_{n,i}}}}{s^g \prod_i (1+T_i s)\prod_i \paren{1 + \frac{2\xi_i}{\omega_{n,i}}s + \frac{s^2}{\omega^2_{n,i}}}}
\end{equation}
con $\mu$ guadagno, $\tau_i, T_i$ costanti di tempo, e ancora smorzamenti e pulsazioni naturali. Cercheremo di spiegare il significato di tutto ciò più avanti.
\end{defin}

Da questa rappresentazione è evidente come sia possibile che tra i due polinomi  \textbf{possano esserci semplificazioni}. Questo si traduce in una riduzione del numero di autovalori, ergo  \textbf{perdita di informazioni sul sistema}, cosa che non avviene nel dominio dei tempi. Spiegare questa cosa nel dettaglio non è particolarmente facile, in quanto ha a che fare con modifiche a livello di raggiungibilità ed osservabilità del sistema. Possiamo dare un'idea concettuale: è possibile che una determinata semplificazione porti alla rimozione di un modo naturale divergente. Questo è un problema! Dal nostro punto di vista il sistema, essendo composto da modi convergenti, lo immaginiamo a sua volta convergente, ma \textit{internamente continueranno ad esistere delle dinamiche divergenti che faranno comunque, alla fine, divergere il sistema}. Il problema è che noi dall'esterno non riusciremo ad accorgercene. Una metafora ideale per la compresione di questo concetto è quella in cui una persona guida una macchina tranquillamente su una strada senza sapere che nel portabagagli c'è una bomba pronta per esplodere; dal punto di vista del conducente il sistema si starà comportando nel modo desiderato, ma noi, sapendo le dinamiche interne dell'auto - la presenza della bomba - abbiamo la certezza che prima o poi esploderà.

%\begin{lemma}
%Uno stato di un sistema si dice raggiungibile se è possibile, mediante opportuna scelta dell'ingresso, condurre verso di esso la traiettoria del sistema in analisi in un tempo finito arbitrario. Un sistema LTI è completamente raggiungibile se e solo se il rango della \textbf{matrice di raggiungibilità}
%\begin{equation*}
%M = \begin{bmatrix}
%B & AB & A^2B & \cdots & A^{n-1}B 
%\end{bmatrix}\in \R^{n\times mn}
%\end{equation*} 
%è pari ad $n$. Nel caso in cui un sistema non sia completamente raggiungibile, è possibile isolare la parte che lo è. Da come abbiamo dato la definzione, la raggiungibilità dipende unicamente dalle matrici $A,B$, dunque dall'equazione di stato.
%\end{lemma}
\begin{esem}
(Semplificazione). Consideriamo il sistema dinamico seguente:
\begin{equation*}
\begin{dcases}
\dot x_1 = -x_1 + x_2  \\
\dot x_2 = -2x_2 + u \\
y=x_2
\end{dcases} \quad \longrightarrow \quad 
\begin{dcases}
\dot x = \begin{bmatrix}
-1 & 1 \\ 0 & -2
\end{bmatrix} x + \begin{bmatrix}
0 \\ 1
\end{bmatrix} u \\
y = \begin{bmatrix}
0 & 1
\end{bmatrix} x + \sparen{0} u
\end{dcases}
\end{equation*}
Passiamo nel dominio di Laplace considerando la funzione di trasferimento (molti passaggi rimossi):
\begin{equation*}
G(s) = C(sI-A)^{-1}B+D = \begin{bmatrix}
0 & 1
\end{bmatrix}\begin{bmatrix}
\frac{s+2}{(s+2)(s+1)} & \frac{1}{(s+2)(s+1)} \\ 0 & \frac{s+1}{(s+2)(s+1)}
\end{bmatrix}\begin{bmatrix}
0 \\ 1
\end{bmatrix} = \frac{\cancel{s+1}}{(s+2)\cancel{(s+1)}} = \frac{1}{s+2} 
\end{equation*}
\end{esem}

\section{Antitrasformazione (sistemi LTI SISO)}
\resource{0.5}{schema}{Schema riassuntivo di quanto visto finora per sistemi LTI}
\bb
Facendo riferimento alla figura in alto ricapitoliamo quanto abbiamo visto in queste pagine: partendo dalla rappresentazione in forma di stato nel dominio dei tempi, siamo riusciti, trasformando, a generare un \textbf{problema immagine} nel dominio di Laplace, composto da equazioni che coinvolgono direttamente la trasformata di stato (non la sua derivata) e di uscita. Mediante la funzione di trasferimento $G(s)$, siamo riusciti poi a dare una relazione esplicita tra ingresso ed evoluzione forzata dell'uscita \eqref{eq:gs_evoforz_y}. 
\begin{defin}{}{}
Recuperiamo ora l'equazione di uscita generale per sistemi LTI, e sostituiamo alla parte forzata proprio l'espressione con la funzione $G(s)$:
\begin{equation*}
Y(s) = C(sI-A)^{-1} x_0 + G(s)U(s)
\end{equation*}
Si può far vedere che \textbf{nel caso SISO} $C(sI-A)^{-1}$ è una matrice $1 \times n$ i cui elementi sono \textit{rapporti di polinomi}. Moltiplicandola per $x_0$ (che, in generale, anche nel caso SISO, è una $n\times 1$), si ottiene uno scalare, come previsto. Anche la $G(s)$ abbiamo visto che, se SISO, si presenta come uno scalare rappresentabile in forma di rapporto di polinomi.  Se, nel corso della trattazione,  considereremo \textbf{solamente ingressi $u(t)$ aventi trasformata $U(s)$ scrivibile come un rapporto di polinomi}  (funzioni gradino, sinusoidali, rampa...), allora l'intera $Y(s)$ sarà scrivibile come un rapporto di polinomi: 
\begin{align}
\label{eq:y_rapp_poli}
\textrm{se sistema SISO e anche $U(s)$ rapp. polinomi} \quad \rightarrow \quad \boxed{Y(s) = \frac{N(s)}{D(s)}}
\end{align}
\end{defin}
\subsection{Sviluppo di Heaviside o in fratti semplici}
A cosa ci servono tutte queste considerazioni sulla struttura di $Y(s)$? Perché vogliamo che si presenti in una certa forma? Preso un problema nei tempi, una volta trovata una sua soluzione immagine all'interno del dominio di Laplace, \textbf{è necessario tornare indietro per riottenere la sua rispettiva nei tempi}! La formula canonica di antitrasformazione non è molto comoda se siamo in casi non riconducibili a quelli elementari. Ci viene dunque in aiuto lo \textbf{sviluppo di Heaviside, o in fratti semplici}.

\begin{defin}{Sviluppo di Heaviside nel caso di poli distinti/semplici - moteplicità 1}{}
Per poter applicare questo metodo, è necessario che la trasformata dell'uscita $Y(s)$ sia \textbf{scrivibile come un rapporto di polinomi}. L'obiettivo dello sviluppo è ottenere una \textbf{fattorizzazione} di $Y(s)$ per ottenere una semplificazione  nel procedimento di antitrasformazione. 

\subsubsection{Caso 1: poli reali}
A partire dalla \eqref{eq:y_rapp_poli}, fattorizziamo il denominatore e scomponiamo quanto ottenuto in una somma di rapporti di polinomi aventi dei \textit{coefficienti} $k_i$, detti \textbf{residui}, al numeratore, e l'i-esimo termine della produttoria al denominatore:
\begin{equation}
\label{eq:heavi_prima}
Y(s) = \frac{N(s)}{D(s)} = \frac{N(s)}{\prod_{i=1}^n (s+p_i)} = \sum_{i=1}^n \frac{k_i}{s+p_i}
\end{equation}
Poiché i poli sono reali, \textbf{i residui associati} saranno a loro volta reali, e si calcolano così:
\begin{equation}
k_i = \eval{(s+p_i)Y(s)}_{s=-p_i} = \eval{(s+p_i)\frac{N(s)}{D(s)}}_{s=-p_i}
\end{equation}
\begin{equation*}
\end{equation*} 
A questo punto si può sfruttare la linearità per l'antitrasformazione, ottenendo:
\begin{equation*}
y(t) = \lat{Y(s)} = \lat{\sum_{i=1}^n \frac{k_i}{s+p_i}} = \sum_{i=1}^n k_i \lat{\frac{1}{s+p_i}}
\end{equation*} 
ma la funzione da antitrasformare è la trasformata della funzione elementare $e^{-\alpha t}1(t)$! Dunque abbiamo che \textbf{in caso di poli reali semplici l'uscita nei tempi è una combinazione lineari di modi naturali esponenziali:}
\begin{equation}
\boxed{y(t) = \sum_{i=1}^n k_i e^{-p_i t}1(t)}
\end{equation}
\subsubsection{Caso 2: poli complessi coniugati}
Similmente a quanto fatto prima, fattorizziamo il denominatore e scomponiamolo. Notiamo una cosa importante: \textbf{questa volta abbiamo poli complessi coniugati}, per cui dovremo considerare delle \textbf{coppie di poli}. Vale che \textbf{anche i residui} sono a loro volta \textbf{complessi coniugati}:
\begin{equation}
Y(s) = \frac{N(s)}{D(s)} = \sum_{i=1}^n \frac{k_{i,1}}{s+p_{i,1}} + \frac{k_{i,2}}{s+p_{i,2}} \quad \textrm{con} \quad \begin{dcases}p_{i,1} = \sigma_i + j\omega_i \\
	p_{i,2} = \sigma_i -j\omega_i
 \end{dcases} \ \textrm{e} \ \begin{dcases}
 k_{i,1} = M_i e^{-j\phi_i} \\
k_{i,2} = M_i e^{j\phi_i}
 \end{dcases}
\end{equation}
A questo punto, antitrasformando e sfruttando la linearità:
\begin{align*}
y(t) & = \lat{Y(s)} = \lat{\sum_{i=1}^n \frac{k_{i,1}}{s+p_{i,1}} + \frac{k_{i,2}}{s+p_{i,2}}} = \sum_{i=1}^n \lat{\frac{k_{i,1}}{s+p_{i,1}} + \frac{k_{i,2}}{s+p_{i,2}}} \\ & = \sum_{i=1}^n \paren{k_{i,1}\lat{\frac{1}{s+p_{i,1}}} + k_{i,2}\lat{\frac{1}{s+p_{i,2}}}} = \sum_{i=1}^n k_{i,1} e^{-p_{i,1}t}1(t)+k_{i,2}e^{-p_{i,2}t}1(t) = (\star)
\end{align*}
Andando a sostituire le espressioni complete di poli e residui e facciamo altri conticini, ricordandoci, in particolare, che la somma di due complessi coniugati dà come risultato due volte la parte reale del numero stesso:
\begin{align*}
(\star) & = \sum_{i=1}^n M_ie^{-j\phi_i} e^{-(\sigma_i + j\omega_i)t}1(t)+M_ie^{j\phi_i} e^{-(\sigma_i - j\omega_i)t}1(t) \\ & = \sum_{i=1}^n M_i e^{-\sigma_it}\underbrace{\sparen{e^{- j(\omega_it + \phi_i)}+e^{j(\omega_it + \phi_i)}}}_{=2\cos(\omega_i t + \phi_i)}1(t) = \sum_{i=1}^n 2M_i e^{-\sigma_it} \cos(\omega_i t + \phi_i)1(t)
\end{align*}
Otteniamo la scrittura finale dell'antitrasformata di $Y(s)$, i.e. una \textbf{somma di esponenziali moltiplicati per una sinusoide}:
\begin{equation}
\label{eq:heavi_compcon}
\boxed{y(t) = \sum_{i=1}^n 2M_i e^{-\sigma_it} \cos(\omega_i t + \phi_i)1(t)}
\end{equation}
\end{defin}
Abbiamo ottenuto un risultato simile a quello visto nella decomposizione modale nel dominio dei tempi, ma con un dettaglio in più molto importante: \textbf{mentre nei tempi avevamo parlato di combinazione lineare di modi naturali senza però fornire un metodo effettivo per il calcolo dei coefficienti che prendevano parte alla combinazione} (probabilmente perché non è banale), \textbf{qui questi, i.e. i residui, vengono esplicitamente calcolati}.
\bb
Ancora una volta, lo spartiacque è l'asse immaginario: i modi naturali suddetti convergeranno, rimarranno costanti, o divergeranno a seconda della loro \textit{parte reale} (risp. negativa, nulla, positiva). 
\bb
\begin{minipage}
{0.5\textwidth}
\resource{0.4}{modi_nat_lapl_reali}{Modi per poli reali distinti}
\end{minipage}
\begin{minipage}
{0.5\textwidth}
\resource{0.4}{modi_nat_lapl_compl}{Modi per poli complessi coniugati distinti}
\end{minipage}
\begin{esem} (Antitrasformazione con sviluppo di Heaviside per poli reali semplici.) Calcolare l'uscita $y(t)$ di un sistema avente funzione di trasferimento $G(s) = \frac{1}{s+p}$, a fronte dell'ingresso $u(t) = 5(t)$ (gradino di ampiezza $5$). Vale la relazione:
\begin{equation*}
Y(s) = G(s) U(s) =  \frac{\lt{5(t)}}{s+p} = \frac{5\lt{1(t)}}{s+p} = \frac{5}{s(s+p)} = \frac{k_1}{s} + \frac{k_2}{s+p} = (\star) 
\end{equation*}
Calcoliamo i residui:
\begin{equation*}
k_1 = \eval{sY(s)}_{s=0} = \eval{\frac{5}{s+p}}_{s=0} = \frac{5}{p} \quad \quad \quad k_2 = \eval{(s+p)\frac{5}{s(s+p)}}_{s=-p} = \eval{\frac{5}{s}}_{s=-p} = -\frac{5}{p}
\end{equation*}
Sostituiamo:
\begin{equation*}
(\star) = Y(s) = \frac{5}{sp} - \frac{5}{p(s+p)} = \frac{5}{p}\paren{\frac{1}{s} - \frac{1}{s+p}}
\end{equation*}
A questo punto possiamo antitrasformare facilmente, tenendo conto che la quantità $5/p$ è un coefficiente:
\begin{equation*}
y(t) = \frac{5}{p}\paren{\lat{\frac{1}{s}} - \lat{\frac{1}{s+p}}} = \frac{5}{p}1(t)-\frac{5}{p}1(t)e^{-pt}.
\end{equation*}
Vediamo come l'uscita è una combinazione lineare di modi esponenziali (la cui tipologia dipenderà dal segno di $-p$). Notiamo come Laplace ci ha dato la possibilità di ottenere l'espressione completa, con tutti i coefficienti definiti.
\end{esem}
\begin{esem}
Consideriamo la seguente funzione di trasferimento, e calcoliamo l'antitrasformata dell'uscita a fronte dell'ingresso $u(t) = 1(t)$.
\begin{equation*}
G(s) = \frac{1}{s^2+6s+109} \quad \rightarrow \quad Y(s) = G(s) U(s) = \frac{1}{s(s^2+6s+109)} = (\star)
\end{equation*}
Prima di procedere, vediamo che la $G(s)$ è stavolta scritta mettendo in evidenza dei \textit{polinomi} al posto delle radici complesse coniugate. In particolare, dunque, facendo riferimento alla prima forma fattorizzata vista qualche pagina fa, è possibile \textbf{evidenziare gli smorzamenti e le pulsazioni naturali} dei poli:
\begin{equation*}
D(s) = s^2 + \underbrace{6}_{2\xi \omega_n}s + \underbrace{109}_{\omega_n^2} \quad \textrm{da cui} \quad \omega_n = \sqrt{109}, \ \xi = \frac{3}{\sqrt{109}}
\end{equation*}
Fatto questo piccolo excursus, procediamo con lo sviluppo di Heaviside. Il polinomio $s^2 + 6s + 109$ ha due radici complesse coniugate:
\begin{equation*}
s_1=-3+10j \quad \quad s_2=-3-10j
\end{equation*}
per cui possiamo fattorizzarlo, e scrivere diversamente la trasformata dell'uscita\footnote{nell'esercizio cerco di scrivere le radici, quando queste si presentano nella forma $(s-radice)$, come $(s+radice)$, in modo da essere coerente con le formule viste per lo sviluppo di Heaviside, che usano il $+$}:
\begin{align*}
(\star) = Y(s) & = \frac{1}{s\sparen{s-(-3+10j)}\sparen{(s-(-3-10j)}} = \frac{1}{s(s+(3-10j))(s+(3+10j))} \\ & =\frac{k_1}{s} + \frac{k_{2,1}}{s+(3-10j)} +\frac{k_{2,2}}{s+(3+10j)} = (\star_2)
\end{align*}
Calcoliamo i residui:
\begin{equation*}
k_1 = \eval{sY(s)}_{s=0} = \eval{\frac{1}{(s+(3-10j))(s+(3+10j))}}_{s=0} = \frac{1}{(3-10j)(3+10j)} = \frac{1}{109}
\end{equation*}
\begin{align*}
\eval{k_{2,1} = (s+(3-10j))Y(s)}_{s=-(3-10j)} = \eval{\frac{1}{s(s+(3+10j))}}_{s=-3+10j} = \frac{1}{(-3+10j)(20j)} = \frac{1}{-200-60j}
\end{align*}
\begin{align*}
k_{2,2} & = \eval{(s+(3+10j))Y(s)}_{s=-(3+10j)} = \eval{\frac{1}{s(s+(3-10j))}}_{s=-3-10j} = \frac{1}{(-3-10j)(-20j)} = \frac{1}{-200+60j}
\end{align*}
Notiamo come i due residui associati alle radici complesse coniugate siano a loro volta coimplessi coniugati, mentre quello associato ad $s$ sia reale.  Visto che vogliamo ricondurci ad un'espressione del tipo \eqref{eq:heavi_compcon}, scriviamo i residui in forma esponenziale:
\begin{equation*}
k_{2,1} \approx 4.789\cdot 10^{-3}e^{j163} \quad \quad k_{2,2} \approx 4.789 \cdot 10^{-3}e^{-j163}.
\end{equation*}
A questo punto riscriviamo $Y(s)$:
\begin{equation*}
(\star_2) = Y(s) = \frac{\frac{1}{109}}{s} + \frac{4.789\cdot 10^{-3}e^{j163}}{s+(3-10j)} +\frac{4.789\cdot 10^{-3}e^{-j163}}{s+(3+10j)}
\end{equation*}
Sfruttiamo poi la linearità e altre proprietà per calcolare l'antitrasformata:
\begin{align*}
y(t) &= \frac{1}{109} \lat{\frac{1}{s}} + \paren{4.789\cdot 10^{-3}e^{j163}} \lat{\frac{1}{s+(3-10j)}} + \paren{4.789\cdot 10^{-3}e^{-j163}}  \lat{\frac{1}{s+(3+10j)}} \\ &= \frac{1}{109} 1(t) + 4.789\cdot 10^{-3}e^{j163}e^{-(3-10j)t}1(t) + 4.789\cdot 10^{-3}e^{-j163} e^{-(3+10j)t}1(t) \\ & = \frac{1}{109} 1(t) + 4.789\cdot 10^{-3} \sparen{e^{-3t+j(10t+163)} + e^{-3t-j(10t+163)}}1(t) \\ & = \frac{1}{109} 1(t) + 4.789\cdot 10^{-3} e^{-3t}\sparen{e^{j(10t+163)} + e^{-j(10t+163)}}1(t)
\end{align*}
\bb
Ricordiamoci che la somma di due complessi coniugati è pari a due volte la loro parte reale:
\begin{equation*}
y(t) = \frac{1}{109} 1(t) + 4.789\cdot 10^{-3} e^{-3t}2\cos(10t+163)1(t).
\end{equation*}
L'uscita in questo caso è data dalla somma di un modo esponenziale associato ad un polo reale semplice del tipo $e^{-pt}$ dove $-p = 0$, quindi è del tipo  \textbf{costante}, e da un modo sinusoidale associato a poli complessi coniugati semplici del tipo $e^{-\sigma t}\cos(\omega t + \phi)$ (lasciamo perdere l'ampiezza $M$ dei residui che moltiplica il tutto, in quanto inifluente per lo studio dell'andamento del modo), dove $-\sigma = -3 < 0$, quindi è del tipo \textbf{convergente}. Abbiamo dunque un'\textbf{uscita complessiva convergente}, e avendo l'equazione esatta, è possibile plottarla su Desmos:
\resource{0.180}{desmos-graph}{Andamento dell'uscita nel tempo}
\bb
Facendo il limite per $t \rightarrow \pinf$ della $y(t)$, che per il teorema del valore finale equivale al calcolo del limite per $s \rightarrow 0$ di $sY(s)$, otteniamo il valore di regime del sistema, ossia quello al quale converge alla fine del transitorio:
\begin{equation*}
\llimit{t}{\pinf}{y(t)} = \llimit{s}{0}{sY(s)} = \frac{1}{109}.
\end{equation*}
Faremo successivamente altre considerazioni sull'andamento dell'uscita, introducendo nuove grandezze.
\end{esem}
\newpage
Sfruttiamo un esempio per derivare lo sviluppo in caso di \textbf{poli multipli, i.e. a molteplicità $>1$.}
\begin{esem} Calcoliamo l'uscita $y(t)$ di un sistema LTI  avente funzione di trasferimento $G(s)$ descritta sotto, a fronte di un ingresso rampa $u(t) = t1(t)$.
\begin{equation*}
G(s) = \frac{2}{s+1} \quad \textrm{e} \quad U(s) = \lat{t1(t)} = \frac{1}{s^2} \quad \textrm{da cui} \quad Y(s) = \frac{2}{s^2(s+1)}.
\end{equation*}
In presenza di radici multiple, è possibile pensare il denominatore come un minimo comune multiplo. Di conseguenza, è necessario trovare e considerare \textbf{tutti i sottomultipli che hanno il denominatore come m.c.m.} In questo caso, i sottomultipli sono $s^2, (s+1)$, ma anche $s$! In altre parole, \textbf{bisogna contare le radici multiple con la loro molteplicità}. Procediamo ora con lo sviluppo (i residui associati a poli multipli avranno un doppio pedice per distinguerli da quelli associati a poli semplici):
\begin{equation*}
Y(s) = \frac{k_{1,1}}{s} + \frac{k_{1,2}}{s^2} + \frac{k_2}{s+1}
\end{equation*}
Togliamo di mezzo il residuo associato al polo semplice, che sappiamo già calcolare: $\boxed{
k_2 = \eval{\frac{2}{s^2}}_{s=-1} = 2}$ 
Come calcolo i residui dei poli multipli? Evidenziamo l'obiettivo: ci interessa \textbf{isolare i termini} $k_{1,1}, k_{1,2}$ all'interno dell'espressione di $Y$, per cui vogliamo che tutti gli altri si semplifichino in modo che non ci rompano. Proviamo a \textbf{usare lo stesso approccio che si usa per i poli semplici applicato però al residuo associato alla molteplicità massima}, in questo caso $k_{1,2}$:
\begin{equation*}
\eval{s^2 Y(s)}_{s=0} = \eval{s^2\paren{\frac{k_{1,1}}{s} + \frac{k_{1,2}}{s^2} + \frac{k_2}{s+1}}}_{s=0} = \eval{sk_{1,1} + k_{1,2} + \frac{s^2k_2}{s+1}}_{s=0} = k_{1,2}
\end{equation*}
Funziona! Tutti i termini fuorché il residuo associato ad $s^2$ si semplificano. Perfetto, possiamo procedere con il calcolo sostituendo ad $Y$ la sua espressione iniziale, come al solito:
\begin{equation*}
\boxed{k_{1,2} = \eval{s^2 Y(s)}_{s=0} = \eval{\frac{2s^2}{s^2(s+1)}}_{s=0} = \eval{\frac{2}{s+1}}_{s=0} = 2} 
\end{equation*}
Ci resta da calcolare il secondo residuo del polo multiplo. Proviamo, anche qui, ad usare lo stesso approccio di sempre:
\begin{equation*}
\eval{sY(s)}_{s=0} = \eval{s\paren{\frac{k_{1,1}}{s} + \frac{k_{1,2}}{s^2} + \frac{k_2}{s+1}}}_{s=0} = \eval{k_{1,1} + \frac{k_{1,2}}{s} + \frac{sk_2}{s+1}}_{s=0} = k_{1,1} + \mathbf{\infty} \ ???
\end{equation*}
No! L'altro residuo ci rompe le scatole perché va all'infinito, in quanto la molteplicità di grado ridotto non riesce ad annullare quella di grado massimo, ovviamente, per cui farà restare sempre qualcosa al denominatore. Proviamo a calcolare la \textbf{derivata prima in $s$ dell'espressione usata per il calcolo del residuo di molteplicità massima}, che abbiamo visto essere uguale all'approccio per poli semplici:
\begin{equation*}
\dv{s}(s^2Y(s)) = \dv{s}(sk_{1,1} + k_{1,2} + \frac{s^2k_2}{s+1}) = k_{1,1} + \frac{2sk_2-s^2k_2}{(s+1)^2} = (\star)
\end{equation*}
Già ad occhio il risultato è promettente, perché la derivata ha eliminato il termine $k_{1,2}$ che prima ci andava all'infinito. Valutiamo ora la derivata in $s=0$:
\begin{equation*}
(\star) = \eval{k_{1,1} + \frac{2sk_2-s^2k_2}{(s+1)^2}}_{s=0} = k_{1,1}
\end{equation*}
Perfetto! Questo è proprio il residuo associato alla molteplicità di grado minore. Possiamo procedere:
\begin{equation*}
\boxed{k_{1,1} = \eval{\dv{s}(s^2Y(s))}_{s=0} = \eval{\dv{s}\frac{2}{s+1}}_{s=0} =  \eval{-\frac{2}{(s+1)^2}}_{s=0} = -2}
\end{equation*}
Abbiamo tutto! Finiamo l'esercizio:
\begin{equation*}
y(t) = \lat{Y(s)} = -2\lat{\frac{1}{s}} + 2\lat{\frac{1}{s^2}} + 2\lat{\frac{1}{s+1}}} =  2(-1+t+e^{-t})1(t)
\end{equation*}
\end{esem}
\begin{defin}{Sviluppo di Heaviside nel caso di poli multipli - molteplicita $> 1$}{}
Vediamo ora come comportarci per eseguire lo sviluppo in presenza di poli di questo tipo.
\subsubsection{Caso 1: poli reali}
Partiamo da una fattorizzazione simile a quella vista in precedenza, con l'unica differenza che adesso i singoli fattori sono elevati ad un generico esponente $n_i$ che ne indica la molteplicità:
\begin{equation}
Y(s) = \frac{N(s)}{D(s)} = \frac{N(s)}{\prod_{i=1}^q (s+p_i)^{n_i}} = \sum_{i=1}^q \sum_{h=1}^{n_i} \frac{k_{i,h}}{(s+p_i)^h}
\end{equation}
La prima sommatoria, che va da $1$ a $q$ include tutti i fattori distinti di $D(s)$; la seconda, che va da $1$ ad $n_i$, include tutte le molteplicità dello specifico (i-esimo) fattore selezionato dalla prima sommatoria. Il pedice $h$ per i residui serve proprio, come abbiamo fatto nell'esempio, ad evidenziare a quale molteplicità di quello specifico fattore è associato.
\bb
Per quanto riguarda i residui, abbiamo visto che, in generale (i.e. per tutti quelli a molteplicità minore della massima), bisogna \textbf{derivare} l'espressione del residuo a molteplicità massima (i.e. $k_{i,n_i}$). In particolare, tanto più basso è l'ordine di molteplicità $h$, tanto più alto è l'ordine di derviazione, che, di conseguenza, è $n_i-h$. A questo va aggiunto un termine fattoriale, che serve per semplificare i coefficienti che compaiono nella derivata in fase di abbassamento dei gradi di $s$:
\begin{equation}
k_{i,h} = \eval{\frac{1}{(n_i-h)!}\dv[n_i - h]{s}\sparen{(s+p_i)^{n_i}\frac{N(s)}{D(s)}}}_{s=-p_i}
\end{equation}
Notiamo che, in caso di molteplicità massima, $h=n_i$ e torniamo all'espressione vista nei poli semplici, in quanto abbiamo $0! = 1$ e derivata zer-esima:
\begin{equation}
k_{i,n_i} = \eval{(s+p_i)^{n_i}\frac{N(s)}{D(s)}}_{s=-p_i}
\end{equation}
A questo punto possiamo ricollegarci all'espressione iniziale di $Y(s)$ per il calcolo dell'antitrasformata, utilizzando la linearità:
\begin{equation*}
y(t) = \lat{Y(s)} = \lat{\sum_{i=1}^q \sum_{h=1}^{n_i} \frac{k_{i,h}}{(s+p_i)^h}} = \sum_{i=1}^q \sum_{h=1}^{n_i} k_{i,h} \lat{\frac{1}{(s+p_i)^h}}
\end{equation*}
Sostituendo l'antitrasformata, otteniamo la relazione finale:
\begin{equation}
\boxed{y(t) = \sum_{i=1}^q \sum_{h=1}^{n_i} k_{i,h} \frac{t^{h-1}}{(h-1)!}e^{-p_it}1(t)}
\end{equation}
L'uscita è una \textbf{combinazione lineare di modi esponenziali moltiplicati per un termine polinomiale} dipendente dal grado di molteplicità. Anche qui, a differenza dello studio modale visto nel dominio dei tempi, sappiamo trovare esattamente l'equazione in quanto abbiamo modo di calcolare i coefficienti (i.e. i residui).
\subsubsection{Caso 2: poli complessi coniugati}
Similmente a quanto fatto prima, scomponiamo il denominatore di $Y$, mettendo in evidenza il fatto che in questo caso lavoriamo con poli complessi coniugati:
\begin{equation*}
Y(s) = \frac{N(s)}{D(s)} = \sum_{i=1}^q \sum_{h=1}^{n_i} \frac{k_{i,h,1}}{(s+p_{i,1})^h} + \frac{k_{i,h,2}}{(s+p_{i,2})^h} \quad \textrm{con} \quad \begin{dcases}
p_{i,1} = \sigma_i + j\omega_i \\
p_{i,2} = \sigma_i - j\omega_i
\end{dcases} \ \textrm{e} \ \begin{dcases}
k_{i,h,1} = M_{i,h}e^{-j\phi_{i,h}} \\
k_{i,h,2} = M_{i,h}e^{j\phi_{i,h}}
\end{dcases}
\end{equation*}
Antitrasformando e sfruttando la linearità:
\begin{align*}
y(t) & = \lat{Y(s)} = \sum_{i=1}^q \sum_{h=1}^{n_i}\paren{k_{i,h,1} \lat{\frac{1}{(s+p_{i,1})^h}} +k_{i,h,2} \lat{\frac{1}{(s+p_{i,2})^h}}} \\ & = \sum_{i=1}^q \sum_{h=1}^{n_i}\paren{k_{i,h,1} \frac{t^{h-1}}{(h-1)!}e^{-p_{i,1}t}1(t) +k_{i,h,2} \frac{t^{h-1}}{(h-1)!}e^{-p_{i,2}t}1(t)} \\ & = \sum_{i=1}^q \sum_{h=1}^{n_i}\frac{t^{h-1}}{(h-1)!}\paren{k_{i,h,1} e^{-p_{i,1}t} +k_{i,h,2} e^{-p_{i,2}t}}1(t) = (\star) 
\end{align*}
Sostituiamo le espressioni complete di poli e residui:
\begin{align*}
(\star) & = \sum_{i=1}^q \sum_{h=1}^{n_i}\frac{t^{h-1}}{(h-1)!}\paren{M_{i,h}e^{-j\phi_{i,h}}e^{-(\sigma_i t + j\omega_i t)} +M_{i,h}e^{j\phi_{i,h}}e^{-(\sigma_i t - j\omega_i t)}}1(t) \\ & = \sum_{i=1}^q \sum_{h=1}^{n_i}\frac{t^{h-1}}{(h-1)!}M_{i,h}e^{-\sigma_i t}\paren{e^{- j(\omega_i t+\phi_{i,h})} +e^{j(\omega_i t + \phi_{i,h})}}1(t)
\end{align*}
Svolgendo la somma di due complessi coniugati e riordinando, otteniamo la relazione finale, i.e. una \textbf{combinazione lineare di esponenziali moltiplicati per una sinusoide e per un polinomio} dipendente dalla molteplicità. Anche qui i coefficienti della combinazione (i.e. i residui, qui meno visibili rispetto al caso reale) sono tutti calcolabili. 
\begin{equation}
\boxed{y(t) = \sum_{i=1}^q \sum_{h=1}^{n_i}2M_{i,h}\frac{t^{h-1}}{(h-1)!}e^{-\sigma_i t}\cos(\omega_i t + \phi_{i,h})1(t)}
\end{equation}
\end{defin}
Lo spartiacque è, anche qui, l'asse immaginario, con l'unica differenza che, \textbf{in caso di parte reale nulla non si ha più una condizione accettabile, ma una divergenza}, data dal termine polinomiale $t^{h-1}$. In caso di parte reale $-\sigma_i$ minore di zero si ha invece sempre convergenza, altrimenti divergenza. Anche qui, l'ampiezza $M_{i,h}$ viene ignorata perché non genera cambiamenti nel comportamento del modo.
\resource{0.5}{poli_multipli_heaviside}{Comportamento dei modi in caso di poli multipli}
\section{Risposte di sistemi elementari SISO}
\subsection{Risposta all'impulso}
Consideriamo l'equazione dell'uscita nel caso di evoluzione forzata (quindi stato iniziale nullo) vista in \eqref{eq:gs_evoforz_y}:
\begin{equation*}
Y(s) = G(s)U(s)
\end{equation*}
 Se consideriamo come ingresso del sistema un impulso in $t=0$, i.e. una delta di Dirac, vale che, essendo la sua trasformata pari ad 1, la trasformata dell'uscita è data direttamente da $G(s)$:
\begin{equation*}
u(t) = \delta(t) \quad \rightarrow \quad Y(s) = G(s) 
\end{equation*} 
cioè che \textbf{la risposta all'impulso} è una combinazione lineare dei modi naturali del sistema LTI SISO descritto da $G(s)$. Ricorda che $G(s)$ può essere riscritta come un rapporto di polinomi $N(s)/D(s)$.

\subsection{Risposta ad un ingresso generico}
Riprendiamo l'equazione della trasformata dell'uscita di un LTI:
\begin{equation*}
Y(s) = C(sI-A)^{-1}x_0 + G(s)U(s)
\end{equation*}
e ricordiamoci che entrambi gli addendi sono rapporti di polinomi. Nel dominio dei tempi avremo:
\begin{equation*}
y(t) = y_L(t) + y_F(t) = y_L(t) + (y_{F,G}(t) + y_{F,U}(t))
\end{equation*}
dove $y_L(t), y_{F,G}(t)$ sono combinazioni lineari di modi naturali del sistema con matrici $A,B,C,D$ (ricordiamo infatti che la $G$ è definita sfruttando queste matrici, vedi \eqref{eq:_trasf}); mentre  il termine $y_{F_U}(t)$ è una combinazione lineare di \textit{modi} presenti nell'ingresso $u(t)$, dovuti alle radici del denominatore di $U(s)$ (vediamo infatti che la risposta forzata è definita come $G(s)U(s)$, e $U(s)$ abbiamo detto che ci piace prenderlo come un rapporto di polinomi).
\starbreak
Dunque mentre, in caso di impulso, la riposta è ricavabile solamente considerando i modi naturali della $G$, in questo caso occorre considerare non solo l'evoluzione libera, in quanto non abbiamo supporto stato iniziale $x_0 = 0$, ma anche il contributo che l'ingresso dà, ossia i modi naturali che introduce.
\subsection{Caso generale}
Riprendiamo la prima forma fattorizzata di $G(s)$ nel caso di sistemi LTI SISO (in quanto in quel caso si ha uno scalare scrivibile come rapporto di polinomi):
\begin{equation*}
%\label{eq:prima_fatt}
G(s) = \frac{\rho \prod_i (s+z_i)\prod_i (s^2+2\zeta \alpha_{n,i}s+\alpha^2_{n,i})}{s^g \prod_i(s+p_i) \prod_i(s^2+2\xi_i \omega_{n,i}s + \omega^2_{n,i})}
\end{equation*}
\begin{defin}{}{}
Raccogliendo quanto studiato finora applicato ai sistemi SISO, a partire dal legame tra trasformata dell'uscita e quella dell'ingresso ($Y(s)=G(s)U(s)$) per il calcolo dell'evoluzione forzata, fino alla prima forma fattorizzata della $G$ (vedi \eqref{eq:prima_fatt}), possiamo concludere che \textbf{posso calcolare la risposta forzata di un sistema dinamico con funzione di trasferimento arbitrariamente complessa sommando le risposte di sistemi elementari di primo e secondo ordine:}
\begin{equation}
Y(s) = G(s)U(s) = \underbrace{\sum_i \frac{k_i}{s+p_i}}_{\textrm{I ordine}} + \underbrace{\sum_i \frac{a_i s + b_i}{s^2 + 2\xi_i\omega_{n,i}s+\omega_{n,i}^2}}_{\textrm{II ordine}}
\end{equation}
Questo grazie al fatto che possiamo usare la \textit{sovrapposizione degli effetti.}
\end{defin}
Sistemi del primo ordine sono quelli caratterizzati da $G(s)$ con poli reali; sistemi del secondo ordine invece sono caratterizzati da $G(s)$ con poli complessi coniugati. Il fatto che quest'ultimo caso si traduca in una scrittura molto più complicata del denominatore deriva unicamente dal fatto che \textbf{si vuole trovare un modo per accorpare le radici complesse coniugate in un'unica equazione}:
\begin{align*}
(s-p_{i,1})(s-p_{i,2}) & = (s-(\sigma_i +j\omega_i))(s-(\sigma_i - j\omega_i)) = (s-\sigma_i -j\omega_i)(s-\sigma_i + j\omega_i) \\ & = s^2 - s\sigma_i +\cancel{js\omega_i} - s\sigma_i + \sigma_i^2-\cancel{j\sigma_i\omega_i} - \cancel{js\omega_i}+\cancel{j\sigma_i\omega_i} - j^2\omega_i^2 \\ & = s^2\underbrace{-2\sigma_i}_{=2\xi_i \omega_{n,i}} s + \underbrace{\sigma_i^2+\omega_i^2}_{=\omega_{n,i}^2}= s^2 + 2\xi_i\omega_{n,i}s+\omega_{n,i}^2 \qed
\end{align*}
\resource{1}{sovr_eff_sistemi_elementari}{Utilizzo della sovrapposizione degli effetti per semplificare lo studio dell'uscita}
\bb
Rendiamoci conto di quanto sia ora facile calcolare \textit{esattamente} l'andamento dell'uscita di un sistema lineare nel dominio dei tempi, rispetto ai risultati ottenuti nella forma di stato, che richiedevano il calcolo di un orrendo integrale di convoluzione. 

\chapter{Risposta al gradino di sistemi elementari}
\begin{prop}
Diamo una definizione di \textit{sistema dinamico}: è un' entità caratterizzata da relazioni tra ingressi ed uscite variabili nel dominio del tempo, in base alle variabili di stato. Un sistema può trovarsi in due stati, sostanzialmente: \textbf{a regime, i.e. in equilibrio, fase in cui le variabili restano costanti al loro valore di regime entro un margine di variazione accettabile (range).} L'altro stato è il \textbf{transitorio}, condizione di \textbf{evoluzione ed assestamento} in cui le variabili mutano fino a quando non tornano in un nuovo stato di equilibrio. Durante la fase oscillatoria vedremo che potrebbero esserci condizioni di \textbf{oscillazioni indesiderate}, ad es. overshoot o undershoot, che aumentano il \textbf{tempo di assestamento (settling time)} tra regime iniziale e regime di arrivo. NB: oscillazioni intorno al valore nominale sono considerate normali.
\end{prop}

\subsection{Esempio di calcolo di $G$ per sistema del primo ordine}
L'equazione della dinamica in questo caso è data dalla seconda legge di Newton: $M\ddot z(t) = -b\dot z(t) + F_m(t)$. La prima quantità è un attrito, la seconda è una motrice, i.e. l'ingresso $u$ che proviene dall'esterno e che dipende da noi. Per la rappresentazione in forma di stato, dobbiamo abbassare il grado di derivazione. Invece di considerare uno stato in $\R^2$ prendendo $\{ x_1 =  z, x_2 = \dot z \}$, che richiederebbe poi due equazioni differenziali $\dot x_1 = x_2, \dot x_2 = \ddot z$, siccome non ci interessa la posizione $z$, possiamoo direttamente prendere:
\begin{equation*}
x = \dot z, \quad u = F_m \quad \rightarrow \quad \begin{dcases}
\dot x = -\frac{b}{M} x+\frac{1}{M}u \\
y = x \quad \textrm{i.e prendo come uscita la velocità}
\end{dcases}
\end{equation*}
Questo è ovviamente un sistema LTI SISO, per cui sappiamo da subito che la funzione di trasferimento sarà uno scalare esprimibile come rapporto di polinomi. E infatti (vediamo che uscirà l'inversa di una matrice $1\times 1$, che si traduce nel calcolo di un determinante di una $0\times 0$ per l'unico cofattore della matrice aggiunta \rarr ricordiamo che \textbf{il determinante di una $0\times 0$ è per definizione pari a $1$!}):
\begin{equation*}
G(s) = C(sI-A)^{-1} B = 1\frac{1}{s+\frac{b}{M}} \frac{1}{M} = \frac{1}{Ms+b}
\end{equation*}
Notiamo che la forma di $G$ coincide con quello che abbiamo detto sui sistemi del primo ordine.
\bb
Richiamiamo la seconda forma fattorizzata della $G(s)$ in forma SISO. Ci servirà adesso in quanto vogliamo fare un'analisi dei sistemi un po' diversa, introducendo delle grandezze che la prima forma  non mostra.
\begin{equation*}
G(s) = \frac{\mu\prod_i (1+\tau_i s) \prod_i \paren{1+\frac{2\zeta_i}{\alpha_{n,i}}s + \frac{s^2}{\alpha^2_{n,i}}}}{s^g \prod_i (1+T_i s)\prod_i \paren{1 + \frac{2\xi_i}{\omega_{n,i}}s + \frac{s^2}{\omega^2_{n,i}}}}
\end{equation*}
\newpage
\section{Sistemi del primo ordine}
Vediamo che per i sistemi del \textbf{primo ordine il parametro che ne determina il comportamento dinamico è la costante di tempo $T_i$}.
Di conseguenza, considerando anche un ulteriore fattore costante $\mu$ denominato \textbf{guadagno}, una funzione di trasferimento tipica di un sistema di questa classe ha una forma del tipo:
\begin{equation}
\label{eq:iord_g}
G(s) = \frac{\mu}{1+Ts}, \quad \mu > 0
\end{equation}
Scegliendo opportunamente il segno della $T$ possiamo generare un comportamento stabile (convergente, i.e. poli a parte reale negativa) o divergenti (parte reale positiva). Per garantire il primo \rarr $T > 0$. Vogliamo studiare la risposta al gradino, quindi prendiamone uno generico (ad esempio quello ad altezza $k$, con $k>0$):
\begin{equation*}
u(t) = k1(t) \rightarrow U(s) = \frac{k}{s} \quad \textrm{da cui} \quad Y(s) = G(s)U(s) = \frac{\mu k}{s(1+Ts)}
\end{equation*}
Eseguiamo lo sviluppo di Heaviside per trovare l'uscita nel dominio dei tempi:
\begin{equation}
y(t) = \mu k (1-e^{-t/T})1(t)
\end{equation}
\textbf{L'uscita $y(t)$ descritta sopra vale per tutti i sistemi del primo ordine scrivibili come \eqref{eq:iord_g}}. Dal teorema del valore finale, abbiamo inoltre che:
\begin{equation}
\llimit{t}{\infty}{y(t)} = \llimit{s}{0}{sY(s)} = \llimit{s}{0}{\frac{\mu k}{1+Ts}} = \mu k = y_\infty
\end{equation}
Chiamiamo $y_\infty$ \textbf{asintotica. Da questa capiamo che l'uscita del sistema tenderà al regime $\mu k$}.

\subsection{Tempo di assestamento (o settling time)}
\begin{defin}{}{}

Ma cosa c'entra in tutto ciò la costante di tempo $T$ associata al polo di cui abbiamo parlato fin dall'inizio? \textbf{$T$ determina la velocità con cui il sistema si avvicina al regime}. Se infatti andiamo a valutare laq \textit{derivata della $y(t)$ in zero}, notiamo che dipende in modo inversamente proporzionale a $T$ stessa. La pendenza è legata a quanto in fretta la curva esponenziale salirà verso il punto $y_\infty$.
\begin{equation*}
\dot y(0) = \eval{-\mu ke^{-t/T}\paren{-\frac{1}{T} 1(t)}}_{0} = \frac{\mu k}{T}
\end{equation*}
Colleghiamo a questa costante di tempo la definizione di \textbf{tempo di assestamento}. Questo, indicato con $T_{a, \epsilon}$, indica il \textbf{tempo tale impiegato dalla $y(t)$ per entrare in una fascia vicina al valore di regime $y_\infty$, del tipo $[y_\infty - \epsilon, y_\infty + \epsilon]$ senza più uscirne}. In simboli:
\begin{equation}
T_{a,\epsilon} \quad \textrm{tempo tale per cui vale} \quad (1-0.01\epsilon)y_\infty \leq y(t) \leq (1+0.01\epsilon)y_\infty \quad \forall t \geq T_{a,\epsilon}
\end{equation}
Per i sistemi del primo ordine, questa ha una formula chiusa:
\begin{equation}
T_{a,\epsilon} = T \ln(\frac{1}{0.01\epsilon}).
\end{equation}
Tipicamente $\epsilon = 5$ (analisi al $5\%$) o $1$ (analisi all'$1\%$).
\end{defin}

Torniamo all'equazione del nostro sistema, e calcoliamo il tempo di assestamento per $\epsilon = 5$. Questo significa capire il \textbf{tempo necessario per $y(t)$ rimanga entro il $5\%$ del valore finale $y_\infty$}:
\begin{equation*}
0.95y_\infty \leq y(t) \leq 1.05 y_\infty
\end{equation*}
Utilizziamo la formula, ottenendo:
\begin{equation*}
T_{a, 5} \approx 3T \quad \textrm{se T} = 1sec  \quad T_{a,5} \approx 3sec
\end{equation*}

Per rimanere entro l'$1\%$ del valore finale, invece, \rarr $T_{a,1} \approx 4.6sec $. Di seguito viene presentato un grafico  MATLAB in cui si evidenzia l'andamento di $y(t)$ (e quindi la sua velocità di avvicinamento al regime) per varie costanti di tempo $T$. Notiamo che, essendo $T$ al denominatore all'interno della derivata calcolata sopra, si ha pendenza maggiore tanto più piccola è. Nel grafico viene evidenziata anche la fascia di assestamento, valutata al $10\%$ per visibilità: $0.9 y_\infty \leq y(t) \leq 1.1y_\infty$: 

\resource{0.75}{y(t)_epsilon_5_primo_ord}{Andamento di $y(t)$ ed evidenziazione della fascia di assestamento per $\epsilon = 10$.}
\bb
Definiamo anche un \textbf{errore a regime} valido per queste tipologie di sistemi:
\begin{equation}
e_{\infty} = \abs{y_\infty - k} = \abs{1-\mu}k
\end{equation}
Infine, vediamo come appare un sistema di questo tipo nel dominio dei tempi, e quindi in forma di stato. La rappresentazione non è unica:
\begin{align*}
G(s) = \frac{\mu}{1+Ts} \quad \rightarrow \quad \begin{dcases}
\dot x = -\frac{1}{T}x + \frac{\mu}{T} u \\ y=x
\end{dcases}, \quad \textrm{con} \ T \textrm{ costante di tempo, } \mu \textrm{ guadagno} 
\end{align*}

\section{Sistemi del primo ordine con uno zero}

\section{Sistemi del secondo ordine con poli complessi coniugati}
In questo caso la funzione di trasferimento $G(s)$ è, compatibilmente a quanto abbiamo detto, di questa forma (nella seconda forma fattorizzata l'equazione di secondo grado in $s$ è identica alla seguente, ma esplicitata in modo diverso):
\begin{equation}
G(s) = \mu \frac{\omega_n^2}{s^2+2\xi\omega_n s + \omega_n^2}, \quad \mu > 0 
\end{equation}
Consideriamo anche qui un ingresso a gradino di altezza $k$, quindi con $U(s) = \frac{k}{s}$. Calcoliamo $Y(s)$:
\begin{equation*}
Y(s) = \mu k \frac{\omega_n^2}{s(s^2+2\xi\omega_n s + \omega_n^2)}, \quad \mu > 0 
\end{equation*}
Portando nel dominio dei tempi, si può far vedere che si ottiene la seguente relazione:
\begin{equation}
y(t) = \mu k \paren{1-Ae^{-\xi \omega_n t}\sin(\omega t + \phi)}1(t), \quad A = \frac{1}{\sqrt{1-\xi^2}}, \ \omega = \omega_n\sqrt{1-\xi^2}, \phi = \arccos(\xi) 
\end{equation}
 da cui la conclusione che \textbf{in genere i sistemi del secondo ordine hanno un'andamento oscillatorio smorzato} dalla presenza di un termine esponenziale. Chiamiamo, infatti, \textbf{$\xi$ coefficiente di smorzamento}, definito $\abs{\xi} < 1$.Chiamiamo invece $\omega_n$, definita $> 0$, \textbf{pulsazione naturale.}

\begin{defin}{}{}
Smorzamento $\xi$ e pulsazione naturale $\omega_n$ sono direttamente \textbf{ricavati dai poli complessi coniugati!} In particolare, guardando la dimostrazione fatta qualche pagina fa dove si prendevano le due radici complesse coniugate e si moltiplicavano fra loro per ottenere la forma polinomiale presente nella prima forma fattorizzata (immediatamente prima dell'inizio di questo capitolo), si ottiene, considerando due radici complesse coniugate $p = \sigma + j\omega$, e $p^* = \sigma - j\omega$: 
\begin{equation}
 \omega_n^2 = \sigma^2 + \omega^2 \rightarrow \boxed{\omega_n = \sqrt{\sigma^2 + \omega^2}} \quad \quad 2\xi \omega_{n} = -2\sigma \rightarrow \xi = -\frac{\sigma}{\omega_n} \rightarrow \boxed{\xi = -\frac{\sigma}{\sqrt{\sigma^2+\omega^2}}}
\end{equation}
\end{defin}
Possiamo vedere che questo vale considerando l'equazione $s^2 + 6s + 109$. Le radici sono $p = -3 + 10j$ e $p^* = -3 -10 j$. Utilizziamole per ricavare $\xi, \omega_n$:
\begin{equation*}
\omega_n = \sqrt{9+100} = \sqrt{109} \quad \quad \xi = -\frac{(-3)}{\sqrt{109}} = \frac{3}{\sqrt{109}}
\end{equation*}
In realtà potevamo trovare questi valori immediatamente, senza prima calcolare le radici. Bastava vedere che l'equazione è del tipo $s^2 + 2\xi\omega_n s + \omega_n^2$, da cui
\begin{equation*}
\omega_n^2 = 109 \rightarrow \omega_n = \sqrt{109} \quad \quad 2\xi\omega_n = 6 \rightarrow \xi = \frac{3}{\sqrt{109}}
\end{equation*}
Il fatto che i valori si trovino fra loro indica che quanto detto è vero.
\starbreak
Torniamo a noi.
Il valore di regime, che abbiamo chiamato $y$ asintotico, vale:
\begin{equation}
\llimit{t}{\infty}{y(t)} = \llimit{s}{0}{sY(s)} = \mu k
\end{equation}
L'equazione di uscita presenta un'andamento \textbf{sinusoidale smorzato con un elevazione dell'ampiezza prima del raggiungimento del regime, chiamata sovraelongazione}:
\resource{0.7}{yt_secord_dettagli}{Grafico di $y(t)$ con grandezze notevoli evidenziate}
\newpage
\subsection{Tempo di assestamento, sovraelongazione ed altre grandezze}
\begin{defin}{}{}
\subsubsection{Tempo di assestamento}
Sebbene la definizione sia identica a quella vista per i sistemi del primo ordine, qui non abbiamo una formula chiusa, diretta per il calcolo. Prendiamo dunque per buono i seguenti risultati, validi per ogni sistema la cui funzione di trasferimento  è scrivibile come ad inizio sezione:
\begin{equation}
T_{a,5} \approx \frac{3}{\xi \omega_n} \quad T_{a,1} \approx \frac{4.6}{\xi \omega_n}
\end{equation}

\subsubsection{Sovraelongazione percentuale}
Indica la \textbf{differenza percentuale tra massima sovraelongazione $y_{max}$ e $y$ asintotica:}
\begin{equation}
S\% = 100 \frac{y_{max} - y_\infty}{y_\infty}
\end{equation}
Questa, per sistemi del secondo ordine descirivibili come sopra, si traduce in 
\begin{equation}
S\% = 100 e^{\frac{-\pi \xi}{\sqrt{1-\xi^2}}}
\end{equation}
\subsubsection{Tempo di salita e di ritardo}
Il primo è definito come il tempo che $y(t)$ impiega per passare dal $10\%$ al $90\%$ del valore finale; il secondo come il tempo impiegato per ottenere il $50\%$ del valore finale.
\end{defin}
\subsection{Considerazioni sulla sovraelongazione}
Dall'equazione della sovaelongazione percentuale si vede come questa, per questa tipologia di sistemi del secondo ordine, \textbf{dipende solamente dallo smorzamento $\xi$}. \resource{0.8}{y(t)_secondo_ordine_smorz}{Comportamento dell'uscita al variare dello smorzamento}
Plottando quell'equazione, inoltre, si ottiene una funzione \textbf{monotona decrescente}. Visto il legame di $\xi, \omega_n$ con le radici, possiamo, dato un valore massimo di sovraelongazione $S^*$, determinare dal grafico il $\xi^*$ corrispondente.
\resource{0.5}{sovr_xi}{Legame tra $S$ e $\xi$}
\bb
Quanto più è alto $\xi$, tanto più bassa sarà $S$, e quindi avremo un segnale molto più valore finale. Notare che se $\xi = 0$ il sistema è in oscillazione perpetua, in quanto si annulla il termine esponenziale che moltiplica il seno all'interno dell'espressione della $y(t)$.

\subsection{Luogo dei punti a tempo di assestamento costante}
L'obiettivo ora è caratterizzare i sistemi del secondo ordine con poli complessi coniugati la cui \textbf{risposta al gradino ha lo stesso tempo di assestamento}. Abbiamo visto che valgono le approssimazioni viste qualche pagina fa:
\begin{equation*}
T_{a,5} \approx \frac{3}{\xi \omega_n} \quad \quad T_{a,1} \approx \frac{4.6}{\xi \omega_n} \quad \quad \textrm{ma } \xi\omega_n  = -\sigma = \Re{p}
\end{equation*}
dunque \textbf{tutti i poli aventi stessa parte reale avranno risposta al gradino con stesso tempo di assestamento.} Il luogo di punti è quindi \textbf{l'insieme delle rette parallele all'asse immaginario.}
\resource{0.5}{luogo}{A poli presi sullo stesso asse immaginario corrisponde uguale tempo di assestamento}
\subsection{Luogo dei punti a sovraelongazione costante}
Se riprendiamo l'equazione della sovraelongazione percentuale:
\begin{equation*}
S\% = 100e^{\frac{-\pi\xi}{\sqrt{1-\xi^2}}}
\end{equation*}
e ricordiamo che abbiamo definito $\phi = \arccos(\xi)$, dove $\phi$ è lo \textbf{sfasamento dei poli} complessi coniugati, concludiamo che il luogo dei punti è \textbf{l'insieme dei complessi aventi stessa fase $\phi$}, i.e. stesso angolo con il semiasse reale sul piano di Gauss. Sono tante \textbf{semirette uscenti dall'origine} degli assi.

\resource{0.5}{sovrael_costante}{A poli presi sulla stessa semiretta all'interno del piano di Gauss corrisponde sovraelongazione costante}
\bb
Le semirette uscenti selezionate sono prese in coppia, una al di sopra dell'asse reale, una al di sotto. Questo perché assocate a poli che sono fra loro complessi coniugati. Di conseguenza, lo sfasamento, in modulo, è identico.

\subsection{Mappatura di specifiche temporali (assestamento, sovraelongazione) nel piano complesso}
Avendo definito dei luoghi di punti sul piano di Gauss dove si possono apprezzare caratteristiche temporali di assestamento e sovraelongazione comuni, possiamo ora \textbf{caratterizzare sistemi del secondo ordine con poli complessi coniugati aventi $S\% \leq S^*$, e $T_{a,5} \leq T^*$.} Queste particolari specifiche 
si traducono nella scelta di uno specifica retta parallela all'asse immaginario, e di una specifica coppia di semirette uscenti dall'origine, ovviamente aventi stesso angolo in modulo rispetto all'asse reale.
\bb
Diamo infine una breve informazione su come un sistema del secondo ordine di questa tipologia può presentarsi nel dominio dei tempi, in forma di stato. Anche questa volta, la rappresentazione di sotto NON è univoca:
\begin{equation*}
G(s) = \mu \frac{\omega_n^2}{s^2+2\xi\omega_n s + \omega_n^2} \quad \rightarrow \quad \begin{dcases}
\dot x_1 = x_2 \\
\dot x_2 = -\omega_n^2 x_1-2\xi\omega_n x_2 + \mu\omega_n^2 u \\
y = x_1
\end{dcases}
\end{equation*}

\section{Sistemi del secondo ordine con poli reali}
Presentano una funzione di trasferimento del tipo:
\begin{equation}
G(s) = \frac{\mu}{(1+T_1s)(1+T_2s)}, \quad \mu, T_1, T_2 > 0 \ \textrm{e} \ T_1 > T_2 \textrm{ senza perdita di generalità}
\end{equation}
La risposta al gradino nel dominio di Laplace (ricorda $U(s) = \frac{k}{s}$), è:
\begin{equation*}
Y(s) = \frac{\mu k}{s(1+T_1 s) (1+T_2s)}
\end{equation*}
Procedendo con Heaviside, otteniamo l'equazione di uscita:
\begin{equation}
y(t) = \mu k\paren{1 - \frac{T_1}{T_1 - T_2}e^{\frac{-t}{T_1}} + \frac{T_2}{T_1-T_2}e^{-\frac{t}{T_2}}}1(t)
\end{equation}
Sebbene i modi presenti siano di tipo esponenziale, analizzando la derivata prima $\dot y(0) = 0$. Analizzando invece $\ddot y(0) = \frac{\mu k}{T_1T_2}$. Guardando però il limite dell'oggetto $y(t)$, troviamo:
\begin{equation}
\llimit{t}{\pinf}{y(t)} = \llimit{s}{0}{sY(s)} = \mu k
\end{equation}
Abbiamo dunque parecchie somiglianze con i sistemi del primo ordine visti all'inizio, sia in termini di tipologia di modi, che per quanto riguarda il valore di regime $y_\infty$. Tuttavia, essendo derivata prima nulla, in zero non troviamo una pendenza immediata. Vediamo meglio con il grafico:
\resource{0.7}{y(t)_sec_ord_poli_reali}{Uscita di un secondo ordine a poli reali, quella tratteggiata è l'uscita di un primo ordine.}
\bb
Procediamo con una caratterizzazione di questi sistemi, basata sulla relazione tra $T_1, T_2$.

\subsection{Caso $T_1 \gg T_2 \rightarrow $ Sistemi a polo dominante}
Riprendiamo l'equazione di uscita:
\begin{equation*}
y(t) = \mu k\paren{1 - \frac{T_1}{T_1 - T_2}e^{\frac{-t}{T_1}} + \frac{T_2}{T_1-T_2}e^{-\frac{t}{T_2}}}1(t)
\end{equation*}
Essendo $T_2$ molto piccola rispetto all'altra, l'esponenziale $e^{-\frac{t}{T_2}}$ si esaurisce molto più velocemente dell'altro (se non ci credi plottalo con Desmos). Inoltre, il rapporto $\frac{T_1}{T_1 - T_2} \approx \frac{T_1}{T-1} \approx 1$, da cui otteniamo:
\begin{equation}
y(t) \approx \mu k \paren{1-e^{-\frac{t}{T_1}}}1(t)
\end{equation}
Questa è proprio l'equazione di uscita di un sistema del primo ordine.
\begin{defin}{}{}
Sistemi a polo dominante, in cui $T_1 \gg T_2$ possono essere \textbf{approssimati ad un sistema del primo ordine}, rendendone più facile lo studio in quanto non dovremo preoccuparci di sovraelongazioni.
\end{defin}
\resource{0.5}{polo_dom}{All'aumentare della distanza tra $T_1, T_2$ aumenta la precisione dell'approssimazione al sistema blu.}
\subsection{Caso $T_1 = T_2 \rightarrow$ poli reali coincidenti}
Abbiamo, in questo caso:
\begin{equation}
G(s) = \frac{\mu}{(1+T_1s)^2} \quad \textrm{da cui} \quad Y(s) = \frac{\mu k}{s(1+T_1s)^2} = \frac{\mu k}{T_1^2}\frac{1}{s\paren{\frac{1}{T_1}+s}^2}
\end{equation}
Risolvendo con Heaviside per poli reali multipli, otteniamo:
\begin{equation*}
Y(s) = \frac{k_1}{s} + \frac{k_{2,1}}{\frac{1}{T_1}+s}} + \frac{k_{2,2}}{\paren{\frac{1}{T_1}+s}^2}
\end{equation*}
Calcoliamo i residui:
\begin{equation*}
k_1 = \eval{\frac{\mu k}{\paren{\frac{1}{T_1}+s}^2}}_{s=0} = \mu k \quad \quad k_{2,2} = \eval{\frac{\mu k}{T_1^2}\frac{1}{s}}_{s=-\frac{1}{T_1}} = -\frac{\mu k}{ T_1}
\end{equation*} 
\begin{equation*}
k_{2,1} = \eval{\dv{s} \paren{\frac{\mu k}{T_1^2}\frac{1}{s}}}_{s=-\frac{1}{T_1}} = \frac{-\mu k }{T_1^2(-\frac{1}{T_1})^2} = -\mu k
\end{equation*}
da cui, antitrasformando:
\begin{align*}
y(t) & = \mu k 1(t) - \mu k\lat{\frac{1}{\frac{1}{T_1}+s}} - \frac{\mu k}{T_1} \lat{\frac{1}{\paren{\frac{1}{T_1}+s}^2}}
\end{align*}
otteniamo finalmente:
\begin{equation}
y(t) = \mu k 	\paren{1 -e^{-\frac{t}{T_1}} - \frac{t}{T_1}e^{-\frac{t}{T_1}}}1(t)
\end{equation}
\textbf{Nota bene:} È possibile procedere con il calcolo del tempo di assestamento per i sistemi del secondo ordine con poli reali $T_{a,\epsilon}$, ma non è banale. Dipenderà però sicuramente da $T_1,T_2$.

\section{Sistemi del secondo ordine con poli reali e zero}

\end{document}





