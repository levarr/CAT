\usepackage[margin = 1in]{geometry}
\usepackage[utf8]{inputenc}
\usepackage[T1]{fontenc}
\usepackage{amsthm, amsfonts, amsmath, amssymb}
\usepackage{tcolorbox}
\usepackage{capt-of, subcaption}
\usepackage[figurename=Fig.]{caption}
\usepackage[font=small,labelfont=bf]{caption}
\usepackage{textcomp}
\usepackage{wrapfig}
\usepackage{hyperref}
\usepackage{fancyhdr}
\usepackage{float}
\usepackage{multirow}
\usepackage{mathtools}
\usepackage{physics}
\usepackage{cleveref}

\graphicspath{ {../attachments/} }

\setlength\parindent{0pt}

\renewcommand{\partname}{Parte}
\renewcommand{\contentsname}{Indice dei contenuti}
\renewcommand{\chaptername}{Capitolo}

\newcommand{\starbreak}{%
    \begin{center}
        $\ast$~$\ast$~$\ast$
    \end{center}
}

\newcommand{\bb}{
	\bigbreak
}

\newcommand{\R}{
	\ensuremath
	\mathbb{R}
}

\newcommand{\llimit}[3]{
	\ensuremath
	\lim_{#1 \rightarrow #2} #3
}

\newcommand{\pinf}{
	\ensuremath
	+\infty
}

\newcommand{\minf}{
	\ensuremath
	-\infty
}

\newcommand{\intreal}[2]{
	\ensuremath
	\int_{\R} #1 \dd{#2}
}

\newcommand{\intnoreal}[4]{
	\ensuremath
	\int_{#1}^{#2} #3 \dd{#4}
}

\newcommand{\ifincases}[1]{
	\ensuremath
	\; \textrm{ se } \; #1
}

\newcommand{\sgn}[1]{
	\operatorname{sgn} \left( #1 \right)
}

\newcommand{\paren}[1]{
	\ensuremath
	\left( #1 \right)
}

\newcommand{\sparen}[1]{
	\ensuremath
	\left[ #1 \right]
}

\newcommand{\resource}[3]{
	\begin{center}
	\includegraphics[scale = #1]{#2}
	\end{center}
	\captionof{figure}{#3}
}

\newcommand{\rarr}{
	$\longrightarrow$
}

\tcbuselibrary{theorems, breakable, skins}

\definecolor{col1}{HTML}{FF7878}
\definecolor{col2}{HTML}{51B5F8}
\definecolor{col3}{HTML}{68E1AA}
\definecolor{col4}{HTML}{B869EA}
\definecolor{col5}{HTML}{FF5500}

\newtcbtheorem[no counter]{defin}{}% environment name
      {
      %attach boxed title to top center={yshift=-3.5mm},
      colback=col4!10, colframe=col4, colbacktitle=col4!75, coltitle=black,
      fonttitle=\bfseries,
      sharp corners=all,
      breakable,
      separator sign none,
      }%
{defin}% label prefix

%\newtheorem{them}{Teorema}[section]
\newtheorem{esem}{Esempio}[section]
%\newtheorem{eser}{Esercizio}[chapter]

\pagestyle{fancy}
\fancyhf{}
\fancyhead[EL]{\nouppercase\leftmark}
\fancyhead[OR]{\nouppercase\rightmark}
\fancyhead[ER,OL]{\thepage}
\fancyfoot[EC,OC]{\nouppercase\leftmark}

\renewcommand{\headrulewidth}{1pt}
\renewcommand{\footrulewidth}{1pt}
